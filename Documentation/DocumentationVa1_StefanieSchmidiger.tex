\documentclass[a4paper,fleqn,english]{book}
%
%---------------------------------------------------------------
%  LaTeX - Template, Grundlagen, Tipps, Vorlagen (Version 09)
%  (c) April 2016, Manuel Wipfli, Stefan Lisibach
%  Hochschule Luzern - Technik und Architektur
%  Abteilung Bautechnik
%---------------------------------------------------------------
%
%
%
%--------------------------------------------------------------------------------------------------------------------
% PREAMBLE
%--------------------------------------------------------------------------------------------------------------------

\usepackage{corefiles/hsluBTmaster13}

\hypersetup{%
pdfcreator={pdflatex},
pdfproducer={LaTeX},
pdftitle={UAV Serial Switch},             %%% Titel der Arbeit UNBEDINGT ANPASSEN!
pdfsubject={VA1},                         %%% Thema (subject) UNBEDINGT ANPASSEN!
pdfauthor={Stefanie Schmidiger},          %%% Autor UNBEDINGT ANPASSEN!
pdfkeywords={Aeroscout},                  %%% Stichwörter UNBEDINGT ANPASSEN!
bookmarksnumbered=true,                   %%% Nummerierte Bookmarks
bookmarksopen=true,                       %%% Bookmarks bei PDF-öffnen bereits geöffnet?
colorlinks=false,                         %%% Farbig markierte Links
plainpages=false,                         %%% zur korrekten Erstellung der Bookmarks
bookmarksopenlevel=1,												 %%% Bookmarks nur bis Hierarchiestufe Section geöffnet
pdfpagelabels,                            %%% zur korrekten Erstellung der Bookmarks
hidelinks,                                %%% Links verstecken
pdfpagelayout=TwoPageRight                %%% Voreingestellte Ansicht im PDF-Editor (z.B. Acrobat)
}%

\graphicspath{{pictures/}}                %%% Pfad, wo die Bilder abgelegt werden
\bibliography{corefiles/literaturDocumentation}        %%% Datei für die Literaturquellen

%\watermark{truefirstpage} % Wasserzeichen "Entwurf": (trueall, truefirstpage,false)


%--------------------------------------------------------------------------------------------------------------------
% DOKUMENT
%--------------------------------------------------------------------------------------------------------------------

%- - - - - - - - - - - - - - - - - - - - - - - - - - - - - - - - - - - - - - - - - - - - - - - - - - - - - - - - - - 
\begin{document}													% Nicht editieren!
\lsstyle                               % Ab hier Zeichenabstand +10 (Nicht editieren!)
\fontsize{10.5}{13.7}\selectfont       % In nachfolgenden Seiten Font 10.5pt (Nicht editieren!)
\pagenumbering{alph}                   % Nötig für Richtigkeit von backref-Verweisen (Nicht editieren!)
%- - - - - - - - - - - - - - - - - - - - - - - - - - - - - - - - - - - - - - - - - - - - - - - - - - - - - - - - - - 


%--------------------------------------------------------------------------------------------------------------------
% TITELBLATT, VERSIONSTABELLE UND SELBSTSTÄNDIGKEITSERKLÄRUNG
%--------------------------------------------------------------------------------------------------------------------

\prestuffmastershort                                 % Funkt. die TB, Versionstab. und Selbstst.-Erkl. generiert
{                                                    % 1. Input kann auch selber noch angepasst werden
%\huge\textbf{\LaTeX}\\                               %%% Titel der Arbeit 1. Zeile 
%\vspace{2mm}
\huge\textbf{Secure and Reliable Data Communication\\
    for Robotics}
%\vspace{8mm}
%\Large\textbf{bla} %%% Untertitel der Arbeit
}
{Vertiefungsmodul II}                         	       %%% Art der Arbeit
{Stefanie Schmidiger}          	      		 			%%% Autor
{Prof. Erich Styger}                          	       		%%% Advisor
{Dr. Christian Vetterli}                                   		%%% Experte
{Horw}                                          		 	%%% Ort
{2018}                                         		 		%%% Jahr XXXX
{15.06.}                                       		 		%%% Tag und Monat der Selbstständigkeitserklärung XX.XX.
{Version 0 & Initial Document & 15.06.18 & Stefanie Schmidiger}    	%%% Änderungsverzeichnis, für neue linie \\



%--------------------------------------------------------------------------------------------------------------------
% VORWORT
%--------------------------------------------------------------------------------------------------------------------

\vorwort%
{true} % ist Vorwort vorhanden? (true,false)
{% Vorwort
%
\todo{this is a copy-paste of va1!! rewrite!!}
With unmanned vehicles, there are always on-board and off-board components. Data transmission between those components is of vital importance. Depending on the distance between vehicle and ground station, different data transmission technologies are ideal. So far, each device was connected to a single modem and the data transmission technology used could not be switched during operation.\\
In a previous project, a serial switch had been designed with four RS-232 interfaces that act as data input and output for devices and four RS-232 interfaces for transmitters and modems. This hardware is very flexible: data routing and transmission behavior is configurable by the user. The application running on the serial switch collects data from connected devices, puts it into a data package and sends it out via the configured transmitter. The corresponding second serial switch receives this package, extracts and verifies the payload, sends it out to the corresponding device and optionally sends an acknowledge back to the package sender.\\
A Teensy 3.2 development board has been used as a micro controller unit. The software was written in the Arduino IDE with the provided Arduino libraries. As the project requirements became more complex, the limit of only a serial interface available as a debugging tool became more challenging. In the end, the software ran with more than ten tasks and an overhaul of the complex structure was necessary.\\
This document describes the refactoring process of the previous project. In the scope of this work, an adapter board has been designed so the previous hardware could be used with the more powerful Teensy 3.5 development board and a hardware debugging interface. A new software concept for the Teensy 3.5 was developed and implemented.\\
The Teensy 3.5 is configured to run with FreeRTOS. The developed software uses the task scheduler and queues of the operating system to provide the same functionalities as the previous software for Teensy 3.2.\\
The new software concept for the Teensy 3.5 is easy to understand, maintainable and expandable. Even though the functionality of the finished project remains the same as in the first version with Teensy 3.2 and Arduino, a refactoring has been necessary. Now further improvements and extra functionalities can be implemented more easily as suggestions are given and issues are reported within this document.} % Text des Vorworts
{Horw, June 2018} % Ort, Datum
{Stefanie Schmidiger} % Verfasser des Vorworts


%--------------------------------------------------------------------------------------------------------------------
% ZUSAMMENFASSUNG
%--------------------------------------------------------------------------------------------------------------------

\zusammenfassung%
{truedeutsch} % Ist Abstract vorhanden?(truebothsamepage, truebothseparatepages, truedeutsch, false)
{% 02_Summary
%
With unmanned vehicles, there are always on-board and off-board components. Data transmission between those components is of vital importance. Depending on the distance between vehicle and ground station, different data transmission technologies are ideal. \\
In a previous project, the hardware for a Serial Switch has been designed that features four RS-232 interfaces to connect data processing and generating devices and four RS-232 interfaces to connect modems for data transmission. The application running on the designed base board assembles data packages with the received data from its devices and sends those data packages out to the modems for transmission. The corresponding second Serial Switch receives those data packages, checks them for validity and extracts the payload to send it out to its devices.\\
A Teensy 3.5 development board acts as the main micro controller and is running with FreeRTOS. Baud rates of the serial interfaces, acknowledges, package resend behavior and other parameters can be configured by altering the config file located on the SD card.\\
The main goal of this project is to add data security and to improve transmission reliability. The software developed in the previous project is taken as a basis, but it needed further work before any new features could be implemented. \\
First, logging, CRC check and other items needed were added. Then, the runtime behavior of the application was analyzed with the Percepio Tracelyzer to ensure sufficient CPU resources left for encryption.\\
Then, data reliability was improved in the following ways:\begin{itemize}
    \item Adding an error correction code to recover original data from faulty packages
    \item Enhancing the resend behavior and supporting transmission of the same package over multiple wireless interfaces
    \item Adding a Smart Wireless Selection algorithm that choses the most reliable wireless channel for package transmission
\end{itemize}
The Golay error correction code was chosen to recover data lost due to interference. The retransmission behavior was enhanced to supports sending the same data over multiple wireless connections and the software was prepared to soon support payload reordering in case the packages get jumbled. A Smart Wireless Selection algorithm was developed that choses the most reliable wireless connection depending on package loss ratio per wireless link and a cost value configurable in the config file.\\
Data security was introduced by adding a crypto acceleration unit to the software project that can later be used to encrypt the communication between two Serial Switches. Suggestions about encryption key storage are given within this document. Integrity of the log files can be guaranteed by adding a hash value to the logged data. The application should thereby not only feed the hashing algorithm with the logged data but also with a secret random number to ensure tempering with the log file is detected. The unique ID of the Teensy 3.5 is suggested to be used as this secret random number.} % Text der Kurzfassung in Deutsch
{} % Text der Kurzfassung in Englisch

%--------------------------------------------------------------------------------------------------------------------
% INHALTSVERZEICHNIS (automatisch generiert)
%--------------------------------------------------------------------------------------------------------------------

%- - - - - - - - - - - - - - - - - - - - - - - - - - - - - - - - - - - - - - - - - - - - - - - - - - - - - - - - - - 
% outsource_TOC
%- - - - - - - - - - - - - - - - - - - - - - - - - - - - - - - - - - - - - - - - - - - - -
\clearpage
\frontmatter      % Beginn der römischen Seitenzahlen
\pagestyle{fancy}       % Für nachfolgende Seiten
\markboth{Inhaltsverzeichnis}{}
\chapter*{Table of Contents} 
\pdfbookmark[1]{Inhaltsverzeichnis}{Inhaltsverzeichnis}
\vspace{-12.8mm}
\makeatletter
\@starttoc{toc}    % Generieren des Inhaltsverzeichnisses
\makeatother
%- - - - - - - - - - - - - - - - - - - - - - - - - - - - - - - - - - - - - - - - - - - - - 	% Nicht editieren!
%- - - - - - - - - - - - - - - - - - - - - - - - - - - - - - - - - - - - - - - - - - - - - - - - - - - - - - - - - - 


%--------------------------------------------------------------------------------------------------------------------
% BEGINN DER NUMMERIERTEN KAPITEL
%--------------------------------------------------------------------------------------------------------------------

%- - - - - - - - - - - - - - - - - - - - - - - - - - - - - - - - - - - - - - - - - - - - - - - - - - - - - - - - - - 
\mainmatter%         	% Nicht editieren!
\pagestyle{fancy}%	% Nicht editieren!
%- - - - - - - - - - - - - - - - - - - - - - - - - - - - - - - - - - - - - - - - - - - - - - - - - - - - - - - - - - 
\chapter{Introduction}
% 023_Introduction
%
This work is being done for Aeroscout GmbH, a company that specialized in development of drones. \\
With unmanned vehicles, there are always on-board and off-board components. Data transmission between those components is of vital importance. Depending on the distance between on-board and off-board components, different data transmission technologies have to be used. \\
So far, each device that generates or processes data was directly connected to a modem. This is fine while the distances between on-board and off-board components does not vary significantly. But as soon as reliable data transmission is required both in near field and far field, the opportunity of switching between different transmission technologies is vital. When data transmission with one modem becomes unreliable, an other transmission technology should be used to uphold exchange of essential information such as exact location of the drone.\\
The goal of this project is to provide a flexible hardware that acts as a switch between devices and modems. Data routing between all connected devices and modems should be configurable and data priority should be taken into account when transmission becomes unreliable.\\
It should be possible to transmit the same data over multiple modems to reduce the chance of data loss for vital information. At receiving side, this case should be handled so the original information can be reassembled correctly with the duplicated data received. In case of data loss or corrupted data, a resend attempt should be started.\\
The configuration should be read from a file on an SD card. This SD card should also be used to store logging data. The system should run with Free RTOS and have a command/shell interface. When no devices are connected, the Free RTOS should go into low power mode.\\
Data loss should be handled and encryption and interleaving should be implemented for data transmission.\\
A hardware should be designed that is ready for field, with a good choice of connectors, small and light weight.\\ 
\\
It was not necessary to start from scratch for this project. Andreas Albisser has already developed a hardware with four RS-232 interfaces to connect different data generating and processing devices and four RS-232 interfaces to connect modems. As a micro controller he used the Teensy 3.2, a small, inexpensive and yet powerful USB development board that can be used with the Arduino IDE.\\
Andreas Albisser also developed a software for the designed UAV Serial Switch base board. The software concept implemented became more complex as the requirements were expanded during development. The finished product did not fulfill all requirements of Aeroscout GmbH. Therefore this follow up project was initiated with new requirements and the hope of a better and easier expandable software as an outcome.\\
Not all requirements can possibly be implemented within one semester, but good ground work should be provided for further modifications and expansions.\\
Because encryption requires a more powerful micro controller than has been used by Andreas Albisser, some hardware modifications are required in the scope of this project. The most profund change is the micro controller and usage of Free RTOS. This will therefore be the main focus inside the project. The aim is to have a stable application with at least as many features and working configuration parameters as the old software had.\\
Some requirements demand hardware changes on the base board so an evaluation needs to be done inside this project to decide how to proceed and where to invest time.\\
A detailed task description can be found in chapter two. An overview and critical analysis of the hardware and software provided by Andreas Albisser is in chapter three. In chapter four, all hardware changes that have been done in the scope of this project are described, followed by chapter four with a description of the software developed. Chapter six is for the conclusion and lessons learned.
%
\chapter{Requirements}%
% Requirements
%
\label{sec:txtAufgabenstellung}
This project has been done for the company Aeroscout GmbH. Aeroscout specialized in the development of drones for various needs. \\
With unmanned aerial vehicles, the communication between on-board and off-board devices is essential and a reliable connection for data transmission is necessary. While the drone is within sight of the control device, data can be transmitted over a wireless connection. With increasing distances, other means of transmission have to be selected such as GPRS or even satellite.\\
So far, the switching between different transmission technologies could not be handled automatically. The data stream was directly connected to a modem and transmitted to the corresponding receiver with no way to switch to an other transmission technology in case of data transmission failure. A visualization of this set up can be seen in \autoref{fig:picOldSetup}.
\spicv{PreviousHwSetup.png}{Previous system setup for data transmission}{\label{fig:picOldSetup}}{100}%
\spicv{NewHwSetup.png}{New system setup for data transmission}{\label{fig:picNewSetup}}{120}%
In the previous project, a flexible platform was developed that acted as a Serial Switch with multiple input and output interfaces for connecting devices and transmitters. See a sample setup in \autoref{fig:picNewSetup}. The hardware has four UART interfaces for devices such as sensors and actors and four UART interfaces for modems. The software developed in the previous project provided basic functionalities such as routing data between devices and modems and retransmission in case of data loss due to interference or an unstable connection. The application was still in its first stage but mostly running stable and provided the basic functionalities correctly.\\
In the scope of this project, the software should first be refactored and all pending requirements from the previous project that are necessary to proceed with the requirements for this project should be implemented. The  refactoring and implementation of pending features includes: \begin{itemize}
    \item Order and assemble at least two Teensy Adapter Boards with silk screen
    \item Overview of task priorities and interrupt priorities
    \item Logging of exchanged data
    \item Improved debug output
    \item Analysis of runtime behavior of application
    \item Analysis of memory usage of application
\end{itemize}
Only when the above mentioned tasks are completed can the next and most important phase of the project be started.\\
The main goal of this project is to implement two more key features: Reliable data exchange and data security. Both items are elaborated in more detail below. They should not only be part of the software but the hardware used should also be analyzed for possible useful settings that enhance the data security and reliability behaviour.
%
%
%
\section{Reliable Data Exchange}
Because unmanned aerial vehicles constantly change their position, transmission is not always reliable. About 10\% - 20\% of the transmitted data are lost and the received data might be corrupted due to interference. The application developed in the scope of this project should take this into account and ensure a reliable data stream.\\
Reliability can be improved in various ways: \begin{itemize}
    \item Recovery of lost bytes by adding redundancy
    \item Retransmission of lost packages
    \item Improving the algorithm that selects the wireless connection to be used
\end{itemize}
Improvements on all of the above mentioned concepts should be made within this project.
%
%
%
\section{Data Security}
Data communication between the two Serial Switches should not be interceptable. This results in the following requirements for security during communication: \begin{itemize}
    \item Ensure that the CPU is working at maximum 15\% capacity during periodic data exchange between on-board and off-board components so that there are enough resources left for encryption
    \item Chose a suitable encryption algorithm that requires little computational power and results in little additional data traffic
    \item Find a solution for encryption key generation and encryption key storage
\end{itemize}
Additionally, data transfer between the two serial switches should be logged, similar to the black box concept known from aviation. This results in the following requirements for logging security: \begin{itemize}
    \item A solution has to be found how tempering of the logged data can either be prevented or detected.
    \item Assure that logging does not use more than 10\% CPU capacity
\end{itemize}%
%
\chapter{Approach}%
% Approach
%
This project builds on the outcome of the project UAV Serial Switch. In this last project, the hardware was designed for the Serial Switch and a software was developed that provides the basic functionalities such as data routing between devices and modems, package transmission on wireless side and the concept of acknowledges and retransmission. For details about the outcome of the last project, please read \autoref{sec:txtSoftwareRefactoring}.\\
The requirements for this follow up project can be found in \autoref{sec:txtAufgabenstellung}.\\
Before improving the software developed in the previous project, the hardware components needed for further software development were ordered and assembled. In the scope of the UAV Serial Switch project, a Teensy adapter board was designed and ordered. But because of manufacturing errors, only one adapter board could be assembled successfully. In order to fully test the application in a real-life setup, two Teensy adapter boards are required. Therefore, more adapter boards were ordered, assembled and tested as a first step within this project.\\
Afterwards, the software developed in the scope of the UAV Serial Switch project was tested with the new hardware. Improvements were made where necessary and new features such as logging were added. Details about the software refactoring can be taken from \autoref{sec:txtSoftwareRefactoring}. Before starting with the implementation of data security and realiability, the runtime behavior of the application was analyzed and assessed (see \autoref{sec:txtSystemAnalysis}). This took up more time than expected because of faulty measurements. It seemed that the CPU was already working at full capacity with no room for the extra traffic that the implementation of security and reliability would cause. Only when finding out that the measurements were faulty due to the extra traffic caused by the analyzation tool could the implementation of the project requirements proceed.\\
Before starting with the implementation of reliability and security, configuration possibilities of the used modems were analyzed (see \autoref{sec:txtModems}).\\ 
Reliability of data exchange was implemented first, starting with an error correcting code. Aeroscout then claimed that their focus lays on retransmission in case of package loss which lead to the testing phase of the error correcting code being cut short.\\
Improvements of the retransmission behavior were done next (see \autoref{sec:txtReliability}), followed by the implementation of encryption (see \autoref{sec:txtSecurity}). Due to a lack of time, both retransmission behavior and encryption could not be implemented fully. Only the concepts and some first implementation steps have been done, elaborations about the detailed implementation and next steps are discussed in this paper.\\
The project plan as implemented can be seen in Appendix \autoref{app:ProjectPlan}.%
%
\chapter{Software Refactoring} \label{sec:txtSoftwareRefactoring}% \
% Software Refactoring
%
In the scope of a previous project (title ``UAV Serial Switch``), the basic functionalites of the Serial Switch were implemented. The end product was working but never tested throughly. Not all features were implemented that are needed for this next project phase. So in order to continue, the application first needed some refactoring and clean up before starting with the implementation of data security and reliability of data exchange.\\
This chapter focuses on the improvements made on the UAV Serial Switch software that was developed in the previous project. As mentioned in the requirements in \autoref{sec:txtAufgabenstellung}, an order for Teensy adapter boards needs to be placed and an overview of the task and interrupt priorities given. The refactoring shall at least include an added logging task and improved debug output.\\
The status of the software and hardware at the start of the project are described in more detail below, followed by the improvements made in the scope of this project.
%
%
%
%
%
%
\section{Status of Previous Project}
In previous projects, the basic functionalities of the Serial Switch were implemented. A hardware was designed and software was written for it. More details about the components provided can be taken from the sections below.
%
%
%
\subsection{Hardware}
The hardware consists of three main components: a baseboard, the main microcontroller board and an adapter board to fit the microcontroller used onto the baseboard.
%
\subsubsection{Baseboard}
\spicv{BareBaseBoard.png}{Baseboard}{\label{fig:picBaseboard}}{100}%
\spicv{HardwareDetails.png}{Block diagram of the components used on the baseboard}{\label{fig:picHardwareDetails}}{100}%
The baseboard has eight serial UART interfaces, four of which are to connect devices such as sensors and actors and four of which are to connect modems for transmission. See \autoref{fig:picNewSetup} for more details. For future references, the side where sensors and actors are connected will be referred to as the device side and the four interfaces for modems will be referred to as the wireless side. The bare baseboard can be seen in \autoref{fig:picBaseboard}. A block diagram about the components used on the baseboard can be seen in \autoref{fig:picHardwareDetails}.\\
The eight user interfaces available are all UART serial interfaces with configurable baud rates. They run on RS232 level, which is +-12V. On device side, there are jumpers available so the user can chose between RS232 input/output and USB input/output for each interface. When the USB is chosen, a serial COM port will appear per USB to UART converter and act as a device input/output.\\
The SPI to UART converter is needed as an interface between the serial interfaces accessible to the user and the microcontroller. It also acts as a hardware buffer that can store up to 128 Bytes of data. There are two hardware buffers on the baseboard, one for the four UART interfaces on device side and one for the four UART interfaces on wireless side. For a detailed description of all hardware components, please read the documentation of the UAV Serial Switch project.
%
\subsubsection{Microcontroller}
\spicv{Teensy35.png}{Teensy 3.5}{\label{fig:picTeensy35}}{80}%
In a first version of the Serial Switch, the Teensy 3.2 development board was used as a main microcontroller unit but was soon replaced by the Teensy 3.5 development board which is still used today. The Teensy 3.5 can be seen in \autoref{fig:picTeensy35}. It features a more powerful microcontroller, more memory and a hardware encryption unit which the Teensy 3.2 does not have. It is therefore more suitable for the implementation of a secure data link, as listed in the requirements (see \autoref{sec:txtAufgabenstellung}).
%
\subsubsection{Adapter Board}
\spicv{TeensyAdapterBoard.png}{Adapter board from Teensy 3.2 to Teensy 3.5 footprint}{\label{fig:picAdapterBoard}}{80}%
Because the baseboard has originally been designed for the less powerful and slightly smaller Teensy 3.2, an adapter board was developed to map the pins of the Teensy 3.5 to the footprint of the Teensy 3.2 development board. The adapter board can be seen in \autoref{fig:picAdapterBoard}.\\
In the scope of the UAV Serial Switch project, the adapter board has been ordered at the interal production at HSLU several times but all boards came back with  multiple production errors (due to no silk overlay). The previous project was finished with only one functional adapter board and more were ordered externally in the scope of this project. The new Teensy adapter boards now have a yellow silk overlay and were assembled and tested successfully.
%
%
%
\subsection{Software} \label{sec:txtSwOutcomeVa1}
\spic{UnsuccessfulPackageTransmission.png}{Package retransmission in case of no acknowledge received}{\label{fig:picPackageTransmission}}
The Teensy 3.5 development board acts as the main microcontroller and runs with the operating system FreeRTOS V9.0.1.\\
The main functionality of the software is data transmission on wireless side. For this, bytes read on device side are collected and put into packages for transmission. Received packages are checked for validity and their payload extracted and pushed out on device side.\\
Acknowledges can be configured to make the data transmission more reliable. If acknowledges are enabled, a package is resent in case no acknowledge is received within the specified timeout. A visualization of this can be seen in \autoref{fig:picPackageTransmission}.\\
\spicv{SwConceptVa1.png}{Simplified software concept at the beginning of this project, showing only the three main tasks}{\label{fig:picSimplifiedSwConceptVa1}}{120}%
\spicv{IsoOsiModel.png}{ISO OSI Model}{\label{fig:picIsoOsiModel}}{60}%
\spic{ExpandedSwConceptVa1.png}{Full software concept at the beginning of this project, showing all tasks}{\label{fig:picFullSwConceptVa1}}%
The software can be configured by modifying the configuration ini file saved on the SD card located on the Teensy 3.5 development board.\\
The main functionality of the software is provided by three tasks, a simplified software concept can be seen in \autoref{fig:picSimplifiedSwConceptVa1}. Task intercommunication is done with queues. Data is pushed onto the queue by one task and popped from the queue by another task for processing. The ISO-OSI model is taken as a reference guide for the responsibilities of each task. The ISO-OSI Model is a representation of a communication standard, as seen in \autoref{fig:picIsoOsiModel}. According to \cite{IsoOsiModel_Wikipedia}, the seven layers have the following responsibilities:\begin{enumerate}
    \item Physical Layer: This layer is responsible for the relationship between the application and the physical transmission medium.
    \item Data Link Layer: This layer is responsible for error detection and correction, determines the protocol to establish and does network layer protocol encapsulation
    \item Network Layer: This layer handles data routing. Message delivery at this layer is not necessarily guaranteed to be reliable
    \item Transport Layer: This layer controls the reliability of a given link, e.g. with acknowledges
    \item Session Layer: This layer provides the mechanism for opening, closing and managing a session between end-user application processes
    \item Presentation Layer: This layer is responsible for the delivery and formatting of information to the application layer for further processing or display
    \item Application Layer: This layer standardizes communication and interfaces used in a network
\end{enumerate}
In the software concept implemented (see \autoref{fig:picSimplifiedSwConceptVa1}), the SPI Handler represents ISO-OSI Layer 1, the Package Handler represents ISO-OSI Layer 2 and the Network Handler represents layer 3 and all upper layers in the ISO-OSI model.\\
Additional tasks such as the ThroughputPrintout task, the Blinky task or the Shell task are for debug purposes only. There is also an Init task which reads the configuration file saved on the SD card, creates all other tasks and afterwards kills itself. A diagram with the full overview of all tasks and their interfacing queues can be seen in \autoref{fig:picFullSwConceptVa1}.\\
Details about the purpose of each task can be taken from below. For a full insight into each task, suggested improvements and implementation details, please consult the documentation of the UAV Serial Switch project.
%
\subsubsection{SPI Handler}
The SPI Handler represents ISO-OSI layer 1 and is the only task that accesses the serial interfaces on the baseboard. It reads data from the SPI to UART converters and pushes it onto the corresponding queue for the next task to process and it pops data from its interfacing queues to push out to the SPI to UART converters.\\
This task does byte handling only and knows nothing about data packages or any other data structures. Data routing is also not done within this task, e.g. bytes popped from the queue for serial interface 3 is also pushed out on serial interface 3.
%
\subsubsection{Package Handler}
The Package Handler pops bytes from the RxWirelessByte queues (interfacing queue to SPI Handler) and assembles them to full data packages. Package validity is checked here by looking at CRC and session number. The successfully assembled and valid packages are then pushed onto the IncomingPackages queue for the next task to process, invalid packages are discarded.\\
The Package Handler also pops packages from the OutgoingPackages queue, adds CRC and session number and disassembles them into bytes to push onto the WirelessTxBytes queue for the SPI Handler to process.\\
Data routing is not done within this task, e.g. packages assembled with bytes from wireless connection 3 are pushed to the IncomingPackages queue for connection 3.
%
\subsubsection{Network Handler}
The Network Handler collects data bytes from the RxDeviceBytes queues, generates packages and pushes them to the OutgoingPackages queue for the Package Handler to disassemble and send out. This task keeps track of the sent packages and received acknowledges. If acknowledges are enabled, resending of packages is done within this task. The Network Handling also does the package routing and extracts the payload from received packages to push down to the TxDeviceBytes queue for the SPI Handler to send out.
%
\subsubsection{Shell}
The Shell task is responsible for the RTT interface. As long as there is an J-Link connection to the target, even with no ongoing debug session, either the RTT Viewer or RTT Client can be started on the connecting computer to access the information provided by the Shell task.\\
The Shell task reads and parses commands supported by the Processor Expert components used. For a list of supported commands, type ``help`` into the RTT terminal. Components that currently support commands from the Shell are:\begin{itemize}
    \item FreeRTOS
    \item FAT File System
    \item Green LED pin
    \item TimeDate component
\end{itemize}
The same terminal of the Shell task is also used by the Throughput Printout task (and all other tasks) to periodically provide information about the performance of the application. For this, the Throughput Printout task pushes debug information onto the interfacing shell queue and upon every execution, the Shell task prints all strings found inside that queue.
%
\subsubsection{Throughput Printout}
This task provides information about the bytes read from and written to the hardware buffer. It also provides information about the packages sent and received on wireless side, bytes lost and general errors and warnings of by the application.\\
All debug information is printed out on the RTT interface and is therefore only available when the Shell task is enabled.
%
\subsubsection{Blinky}
This task periodically toggles the green LED on the baseboard.
%
\subsubsection{Init} \label{subsec:txtInitTask}
This task is the only task created upon startup of the application and runs as soon as the scheduler is started. It reads the configuration file on the SD card and fills the global config variable that is later accessed by all other tasks. Only when the configuration file is read and the content of the config variable verified does this task create all other tasks and afterwards kills itself.\\
It is not possible to read the SD card without the scheduler being started beforehand as the Init task accessed the FatFs file system which requires the scheduler to be running. Furthermore, because all other tasks access the global config variable which is filled within the Init task, the other tasks are created and started later by the Init task.
%
\subsubsection{Idle}
This task is running when no other task is executing. It is provided by the FreeRTOS and no changes have been made to it. It is only listed here for completeness and for the reader to have a full overview of all tasks running.
%
\subsubsection{Task Priorities} \label{subsec:txtTaskPriorities}
The application consists of the tasks mentioned above where each task runs with the following priority:
\begin{itemize}
    \item SPI Handler:			\tab tskIDLE\_PRIORITY+3
    \item Package Handler:		\tab tskIDLE\_PRIORITY+2
    \item Network Handler:		\tab tskIDLE\_PRIORITY+2
    \item Shell:				\tab tskIDLE\_PRIORITY+1
    \item Throughput Printout:	\tab tskIDLE\_PRIORITY+1
    \item Blinky:				\tab tskIDLE\_PRIORITY+1
    \item Init:					\tab tskIDLE\_PRIORITY+2
    \item Idle:					\tab tskIDLE\_PRIORITY
\end{itemize}
Logically high priority tasks have a high priority number and logically low priority tasks have a low priority number. The maximum priority is configurable in the FreeRTOS Processor Expert component and is set to 6. The highest priority task can therefore run with priority tskIDLE\_PRIORITY+5, because the lowest priority number is 0 which is tskIDLE\_PRIORITY. Only the Idle task should be running with tskIDLE\_PRIORITY as it is the fallback task for the scheduler when no other task is ready for execution.
%
\subsubsection{Interrupt Priorities}
The FreeRTOS used allows for 16 interrupt priority levels, with 0 being the logically highest priority and 15 being the logically lowest priority. These possible priorities are have been implemented with four bits, resulting in the following possible interrupt values:
\begin{itemize}
    \item Hexadecimal: 0x00, 0x10, 0x20, 0x30, 0x40, 0x50, 0x60, 0x70, 0x80, 0x90, 0xA0, 0xB0, 0xC0, 0xD0, 0xE0, 0xF0
    \item Decimal: 0, 16, 32, 48, 64, 80, 96, 112, 128, 144, 160, 176, 192, 208, 224, 240
\end{itemize}
These values map to priority level 0 to 15.\\
Negative interrupt priorities are defined by ARM and are part of the core, so they cannot be used. Interrupt priorities with values >= 0 are typically used for devices like UART, SPI etc.\\
The following components within this project use hardware interrupts:
\begin{itemize}
    \item RNG: Random Number Generator, used to create session number for Package Handler task
    \item SPI: Serial interface, used by SPI Handler task to communicate with hardware buffers
    \item SPI2: Serial interface, used by FatFs which is again used by the Logging task to store data on the SD card 
    \item RTC: Real Time Clock, used by the FatFS but no RTC hardware component or battery attached so the RTC is not consistend across reboots and power outs
\end{itemize}
The interrupt priorities set inside this application can be see in the file Vector\_Config.h and have the following values:
\begin{itemize}
    \item RNG: 112 or 0x70
    \item SPI: 48 or 0x30
    \item SPI2: 112 or 0x70
    \item RTC: 112 or 0x70
\end{itemize}
%
\subsubsection{Known Issues and Missing Features} \label{subsec:txtIssuesAndMissingFeatures}
Detailed information about the issues with the provided application can be taken from the documentation of the previous project UAV Serial Switch. Major issues that need to be fixed or added are the following: \begin{itemize}
    \item Packages are not numbered consecutively so there is no way for the receiver to know if a package is missing.
    \item Package reordering is not implemented, the payload of each received package is pushed out immediately. If packages are not received in correct order, e.g. because a package was lost and then retransmitted but in the meanwhile a newer package arrived, the payload sent out on device side is in a wrong order and may therefore be useless for the receiver.
    \item The CRC check of both header and payload is implemented but still commented out because of some issues. Currently, with CRC check enabled, most packages get discarded because of a faulty header. The cause of this problem is a currently unknown software bug.
    \item It is not possible to transmit redundant data over several modems because only packages are numbered and the payload is not. The receiver would not know when the received payload is redundant and could be discarded because package numbering is handled per wireless connection and the same package receives different package numbers when being sent out over multiple modems.
    \item Logging has not been implemented yet.
\end{itemize}
%
\subsection{Conclusion}
With two Teensy adapter boards successfully assembled and tested, work on the software can now be started. In this section, an overview of all interrupt and task priorities has been given.
%
%
%
%
%
%
%
%
%
\section{Added Features} \label{sec:txtAddedFeatures}
\spic{ExpandedSwConceptVa2_BeforeOverhaul.png}{Software concept after added features}{\label{fig:picExpandedSwConceptVa2_BeforeOverhaul}}
Before starting with the implementation of data security and reliability of data exchange, the current software needs to undergo refactoring and the features mentioned above (\autoref{subsec:txtIssuesAndMissingFeatures}) need to be added. The resulting new software concept looks as in \autoref{fig:picExpandedSwConceptVa2_BeforeOverhaul}. Details about the added features can be found below.
%
%
\subsection{Logging} \label{subsec:txtLoggingTask}
All exchanged data has to be logged. This was in fact part of the requirements for the previous UAV Serial Switch project but could not be implemented due to a lack of time. Therefore it was implemented in the scope of this project.\\
A dedicated task was added that handles data logging only. The Package Handler assembles packages out of received bytes and pushes the successfully assembled and valid packages to a queue for the Network Handler to process. When logging is enabled, the Package Handler also creates a copy of the assembled package and pushes it to the IncomingPackages queue of the Logger.\\
The Network Handler puts bytes from the device byte stream into packages and pushes those packages onto the OutgoingPackages queue of the Package Handler for further processing. When logging is enabled, a copy of the package is created and pushed onto the OutgoingPackages queue of the Logger task.\\
Logging can be enabled by setting the LOGGING\_ENABLED parameter to 1 in the configuration file located on the SD card. There is also a LOGGER\_TASK\_INTERVAL parameter that determines the execution period of this task.\\
When logging is enabled, each package in the interfacing queues is converted to a comma separated values string and written into the correct log file. The log files are named accordingly, e.g. all packages in the OutgoingPackages queue for wireless connection 2 are logged into the file txPakWl2.log and all packages in the IncomingPackages queue for wireless connection 1 are logged into the file rxPakWl1.log.\\
Upon startup of the application, the Logger opens all existing log files, moves to the end of the files and adds the header. On every periodic task execution, the Logger task will pull packages from its interfacing queues, convert them into log strings and write them to the SD card. A sample log file could look as follows:
%
\begin{lstlisting}
PackageType;DeviceNumber;SessionNumber;PackageNumber;PayloadNumber;PayloadSize;CRC8_Header;Payload;CRC16_Payload

01;00;45;01;01;0011;88;FE092DFF0000000000000608C004039640;AD7C
01;00;45;02;02;0011;88;FE0930FF0000000000000608C0040371F4FE022EFF00150100;1245
01;00;45;03;03;0011;88;FE0931FF0000000000000608C004039B8A;63E4
01;00;45;04;04;0011;88;FE0932FF0000000000000608C00403A509;A3F6
01;00;45;05;05;0011;88;FE0933FF0000000000000608C004034F77FE2134FF004C0000;FD86
01;00;45;06;06;0011;88;FE0935FF0000000000000608C004032279;B5D3
01;00;45;07;07;0011;88;FE0936FF0000000000000608C004031CFA;75C1
01;00;45;08;08;0011;88;FE0937FF0000000000000608C00403F684;35CF
01;00;45;09;09;0011;88;FE0938FF0000803F0000000000000000000000000000000000000000;7822
01;00;45;0A;0A;0011;88;FE0939FF0000000000000608C00403F865;1985
01;00;45;0B;0B;0011;88;FE093AFF0000000000000608C0040C6E6;D997
\end{lstlisting}
%
The log file output can easily be imported into an Excel file, the values separated and visualized.\\
When the microcontroller writes data to the SD card, the write process does not necessarily take place right away. The main microcontroller on the Teensy has an internal buffer of 512 Bytes and only transfers the content of that buffer down to the SD card when this buffer is full. This write process can be forced by executing a FatFs flush command on the Teensy which results in the internal buffer content being written to the SD card immediately.\\
To ensure that log data is saved on the SD card periodically and not only when the internal buffer is full, the interval at which the flush command is executed can be specified with the parameter SD\_CARD\_SYNC\_INTERVAL (in seconds) in the configuration file.\\
The task interval at which elements inside the interfacing queues (OutgoingPackages and IncomingPackages) are processed and converted to a log string can be specified with the parameter LOGGER\_TASK\_INTERVAL in the configuration file on the SD card. An example of a full configuration file (including a description of all parameters) can be found in the Appendix \autoref{app:txtConfigFile}.
%
\subsubsection{Next Steps for Logging Task}
The standard SD (Secure Digital) format was used inside this application for file access and storage on the memory card. According to \cite{SD_Wikipedia}, SD uses a 9-pin connector as an interface are there are three transfer modes supported:\begin{itemize}
    \item SPI mode 
    \item One-bit SD mode with separate command and (bidirectional) data channel and a proprietary transfer format
    \item Four-bit SD mode (uses extra pins plus some reassigned pins) to support four bit wide parallel transfers. The commands are the same as with one-bit SD mode
\end{itemize}
All 9 pins of the memory card slot are routed to the main microcontroller on the Teensy 3.5. For communication, the SPI mode is used because there is a byte-oriented hardware SPI component available on the microcontroller. This is the fastest mode for the Teensy because with the one-bit or four-bit SD mode, bits would have to be manually transferred one at a time in software (bit-banging) and bidirectionality of the data pins would have to be handled in software as well.\\
Also, the SPI-bus interface mode is the only type that does not require a host license for accessing SD cards.\\
Instead of using the SD format, the application could use the SHDC format. SHDC is an extension of SD, with the same physical connections and dimensions but an increased storage capacity and a FAT32 filesystem instead of FAT12 or FAT16 as supported by SD. According to \cite{SD_SiliconLabs}, SD supports data transfer speeds of 0.9 - 20MB/s while the SHDC supports data transfer speeds of 2 - 40MB/s.\\
Because there was no ready to use Processor Expert component available for SHDC, SD was used. In a later project, the SD component should be replaced with the faster SHDC component to reduce CPU time of the logging task.
%
\subsection{CRC}
The bug with the faulty CRC check has been found and fixed. Both header and payload CRC are now checked inside the Package Handler task. If a package is faulty, it is discarded by this task.
%
\subsection{Debug Output}
The periodic debug text is calculated and assembled by the Throughput Printout task, pushed onto the Shell queue and printed out to the terminal upon the next execution of the Shell task. By adjusting the THROUGHPUT\_PRINTOUT\_TASK\_INTERVAL (in seconds), the frequency of the debug output can be set. \newpage
\begin{lstlisting}
*******************************************************************************************
Device 0 ------>   0 B/s -------> ================= ------->   0 B/s   -------> Wireless 0
                                  ||             ||            0 Datapack/s
Device 1 ------>   0 B/s -------> ||             || ------->   0 B/s   -------> Wireless 1
                                  ||             ||            0 Datapack/s
Device 2 ------>   0 B/s -------> ||             || ------->   0 B/s   -------> Wireless 2
                                  ||             ||            0 Datapack/s
Device 3 ------>   0 B/s -------> ================= ------->   0 B/s   -------> Wireless 3
                                                               0 Datapack/s


Device 0 <------   0 B/s <------- ================= <-------   0 B/s   <------- Wireless 0
                                  ||             ||            0 Datapack/s
Device 1 <------   0 B/s <------- ||             || <-------   0 B/s   <------- Wireless 1
                                  ||             ||            0 Datapack/s
Device 2 <------   0 B/s <------- ||             || <-------   0 B/s   <------- Wireless 2
                                  ||             ||            0 Datapack/s
Device 3 <------   0 B/s <------- ================= <-------   0 B/s   <------- Wireless 3
                                                               0 Datapack/s
                                                               
NetworkHandler: Total number of dropped packages per device input: 0,0,0,0
NetworkHandler: Total number of dropped acknowledges per wireless input: 0,0,0,0
PackageHandler: Total number of invalid packages per wireless input: 0,0,0,0
SPI Handler: Total number of dropped bytes per device byte input: 0,0,0,0
SpiHandler: Total number of dropped bytes per wireless byte input: 0,0,0,0

*******************************************************************************************
\end{lstlisting}
The ThroughputPrintout task does its calculations with float point arithmetics. If CPU occupancy becomes an issue, all variables should be changed to integers so the calculations are less resource-intensive for the microcontroller.
%
%
\subsection{Package Numbering / Payload Numbering}
In the previous implementation, the packages were not numbered consecutively but instead the system tick time (which is the software running time) was used as a package number. There was no way for the receiver to know if a package was missing.\\
A continuously incrementing package number is now used that replaces the previous system time inside the package header. Each wireless connection now has a separate package counter.\\
The package header was also expanded by a payload counter because it was not possible to send the same package out on multiple wireless interfaces and discard all redundant packages on receiving side. The receiver would not know if a package was duplicated because of the separate package numbering per wireless connection. Therefore, a payload counter was introduced as well. Now the Network Handler task can not only handle resending of unacknowledged packages and do payload reordering (thanks to the package number), but it can also discard duplicated packages because each device byte stream now has its own continuous payload counter. The wireless package structure now looks as follows: \newpage
\begin{lstlisting}
/*! \enum ePackType
*  \brief There are two types of packages: data packages and acknowledges.
*/
typedef enum ePackType
{
    PACK_TYPE_DATA_PACKAGE = 0x01,
    PACK_TYPE_REC_ACKNOWLEDGE = 0x02
} tPackType;


/*! \struct sWirelessPackage
*  \brief Structure that holds all the required information of a wireless package.
*  Acknowledge has the same packNr & devNum in header as the package it is acknowledging.
*  The individual packNr for ACK package is in its payload.
*/
typedef struct sWirelessPackage
{
    /* --- header of package --- */
    tPackType packType;
    uint8_t devNum;
    uint8_t sessionNr;
    uint16_t packNr;
    uint16_t payloadNr;
    uint16_t payloadSize;
    uint8_t crc8Header;
    /* --- payload of package --- */
    uint8_t* payload;
    uint16_t crc16payload;
} tWirelessPackage;
\end{lstlisting}
For simplicity, a stripped version of the structure is shown. In the software project, the wireless package structure also holds variables to store the retransmission behavior of a particular package.
%
\subsubsection{Next Steps for Package Numbering / Payload Numbering}
Currently, the application allocates two bytes in the package header for package numbering and two bytes for payload numbering. It suffices to use one byte per parameter as the software will not need keep track of more than 256 packages before the counter restarts at zero. This should be changed in the next software revision to keep the package header as small as possible.
%
%
\subsection{Payload Reordering}
Now that consecutive payload numbering has been introduced, there are different ways on how to handle packages received in wrong order:
\begin{itemize}
    \item \textbf{Payload number ignored:} the receiver does not process the payload number, the payload is extracted from received packages and sent out in the same order as it was received. Redundant payloads (a result of the same package being sent out over multiple wireless connections) will not be detected.
    \item \textbf{Payload reordering:} There is an internal array where received payload is stored for reordering. Together with the configuration parameter PACK\_REORDERING\_TIMEOUT, it determines the reordering behavior of the application. With PACK\_REORDERING\_TIMEOUT, the user can specify how long a missing payload is waited for before this package is skipped and the payload of the next internally stored package is sent out.
    \item \textbf{Only send out new payload:} There is no internal reordering of received payload but when packages get jumbled, the Network Handler task will discard packages with payload numbers older than the one previously sent out on device side.
\end{itemize}
All payload reordering is being done by the Network Handler task.
%
\subsubsection{Next Steps for Payload Reordering}
If the application is configured to ignore the payload number of send payload out only if it is new, then it is running correctly. The ground work has been done for payload reordering but it cannot be used yet because it does not work correctly. For implementation details and status of payload reordering, see \autoref{sec:txtRetransmission}. In a next project phase, more thought should be put into this issue.
%
%
\subsection{Package Transmission Mode} \label{subsec:txtPackageTransmissionMode}
When acknowledges are configured on a wireless connection, there are two ways on how package transmission can be handled:
\begin{itemize}
    \item \textbf{Synchronous transmission:} The next package is only sent out when the previous package has been acknowledged
    \item \textbf{Asynchronous transmission:} Packages are sent out continuously without waiting on the acknowledge of the previous package
\end{itemize}
Synchronous transmission can be enabled for each wireless interface with the parameter\\ 
SYNC\_MESSAGING\_MODE\_ENABLED\_PER\_WL\_CONN. If no acknowledge is configured for a wireless connection, this parameter is ignored on that interface.
%
\subsubsection{Next Steps for Package Transmission Mode}
There is no timeout implemented on this parameter. If a package is never acknowledged, that wireless channel will infinitely be blocked. A timeout should be implemented and the behavior tested with both packages that are acknowledged only after being resent and packages that are never received correctly and therefore never acknowledged.
%
%
\subsection{Static Memory Allocation}
Previously, memory for all tasks and queues was allocated dynamically. Now, memory allocation for tasks and queues is done statically by default. Dyanmic memory allocation can be reenabled inside the FreeRTOS Processor Expert component.\\
With static memory allocation, only the Init tasks still uses dynamically allocated memory because it later deletes itself. All other tasks including their interfacing queues allocate their memory statically.
%
%
\subsection{Conclusion}
Adding missing features is a very time consuming task. There is still lots of work to be done and most new features were implemented in a very basic manner. The functionality of the overall software has improved but should not be used in a end product as it is. Time should be invested in refactoring the application again.\\
The current application is now ready to support payload reordering in case packages are not received in correct order. Both consecutive package and payload numbering have been introduced and a logger task has been added. The analysis of the runtime behavior of the logger (as requested in \autoref{sec:txtAufgabenstellung}) will be subject of \autoref{sec:SystemAnalysis}.%
%
\chapter{System Analysis} \label{sec:txtSystemAnalysis}
% System Analysis
%
Before expanding the application and implementing more features, the system performance needs to be analyzed to ensure sufficient capacity for error correcting codes and encryption.\\
In the previous project, SEGGER SystemView was used to analyze the runtime behavior and CPU load of each task.\\
SEGGER SystemView is a real-time recording and visualization tool for embedded systems that reveals information about runtime behavior of an application. SystemView can  track down inefficiencies and show unintended interactions and resource conflicts.\cite{SeggerSystemView}.\\
In the scope of this project, Percepio Tracealyzer was used instead of SEGGER SystemView because it provides a more in-depth insight into the runtime behavior of the software. The Tracealyzer not only shows the task execution times and RTOS events, it also shows this information in interconnected views and collects data about the CPU load and memory usage.\\
There is both a Processor Expert component for the SEGGER SystemView and one for Percepio Tracelyzer. These components are configured and enabled so that the corresponding code is generated. To use either one of these components, they have to be enabled in the FreeRTOS Processor Expert component. Only then will the debugger provide runtime information to the correct tool.\\
%
%
\section{SEGGER SystemView}
The application developed in the scope of this project runs with four main tasks that perform the main functionality of the software. Task intercommunication is done with queues, where one task always pushes data to a queue and another task pops this data from the queue to process it. This results in thousands of queue operations each second when the UAV Serial Switch is busy.\\
Queue operations are part of the FreeRtos. Because SystemView logs all FreeRtos calls, the additional traffic caused by SystemView when the software is already working to capacity can cause the application to crash. This can be prevented by disabling the logging of queue operations for SystemView. Simply comment the following code lines out in the file SEGGER\_SYSTEM\_VIEW\_FreeRTOS.h (can be found in the GeneratedCode folder):\\
\begin{lstlisting}
//#define traceQUEUE_PEEK( pxQueue )                                    SYSVIEW_RecordU32x4(apiID_OFFSET + apiID_XQUEUEGENERICRECEIVE, SEGGER_SYSVIEW_ShrinkId((U32)pxQueue), SEGGER_SYSVIEW_ShrinkId((U32)pvBuffer), xTicksToWait, xJustPeeking)
//#define traceQUEUE_PEEK_FROM_ISR( pxQueue )                           SEGGER_SYSVIEW_RecordU32x2(apiID_OFFSET + apiID_XQUEUEPEEKFROMISR, SEGGER_SYSVIEW_ShrinkId((U32)pxQueue), SEGGER_SYSVIEW_ShrinkId((U32)pvBuffer))
//#define traceQUEUE_PEEK_FROM_ISR_FAILED( pxQueue )                    SEGGER_SYSVIEW_RecordU32x2(apiID_OFFSET + apiID_XQUEUEPEEKFROMISR, SEGGER_SYSVIEW_ShrinkId((U32)pxQueue), SEGGER_SYSVIEW_ShrinkId((U32)pvBuffer))
//#define traceQUEUE_RECEIVE( pxQueue )                                 SYSVIEW_RecordU32x4(apiID_OFFSET + apiID_XQUEUEGENERICRECEIVE, SEGGER_SYSVIEW_ShrinkId((U32)pxQueue), SEGGER_SYSVIEW_ShrinkId((U32)pvBuffer), xTicksToWait, xJustPeeking)
//#define traceQUEUE_RECEIVE_FAILED( pxQueue )                          SYSVIEW_RecordU32x4(apiID_OFFSET + apiID_XQUEUEGENERICRECEIVE, SEGGER_SYSVIEW_ShrinkId((U32)pxQueue), SEGGER_SYSVIEW_ShrinkId((U32)pvBuffer), xTicksToWait, xJustPeeking)
//#define traceQUEUE_RECEIVE_FROM_ISR( pxQueue )                        SEGGER_SYSVIEW_RecordU32x3(apiID_OFFSET + apiID_XQUEUERECEIVEFROMISR, SEGGER_SYSVIEW_ShrinkId((U32)pxQueue), SEGGER_SYSVIEW_ShrinkId((U32)pvBuffer), (U32)pxHigherPriorityTaskWoken)
//#define traceQUEUE_RECEIVE_FROM_ISR_FAILED( pxQueue )                 SEGGER_SYSVIEW_RecordU32x3(apiID_OFFSET + apiID_XQUEUERECEIVEFROMISR, SEGGER_SYSVIEW_ShrinkId((U32)pxQueue), SEGGER_SYSVIEW_ShrinkId((U32)pvBuffer), (U32)pxHigherPriorityTaskWoken)
#define traceQUEUE_REGISTRY_ADD( xQueue, pcQueueName )                SEGGER_SYSVIEW_RecordU32x2(apiID_OFFSET + apiID_VQUEUEADDTOREGISTRY, SEGGER_SYSVIEW_ShrinkId((U32)xQueue), (U32)pcQueueName)
#if ( configUSE_QUEUE_SETS != 1 )
// #define traceQUEUE_SEND( pxQueue )                                    SYSVIEW_RecordU32x4(apiID_OFFSET + apiID_XQUEUEGENERICSEND, SEGGER_SYSVIEW_ShrinkId((U32)pxQueue), (U32)pvItemToQueue, xTicksToWait, xCopyPosition)
#else
#define traceQUEUE_SEND( pxQueue )                                    SYSVIEW_RecordU32x4(apiID_OFFSET + apiID_XQUEUEGENERICSEND, SEGGER_SYSVIEW_ShrinkId((U32)pxQueue), 0, 0, xCopyPosition)
#endif
#define traceQUEUE_SEND_FAILED( pxQueue )                             SYSVIEW_RecordU32x4(apiID_OFFSET + apiID_XQUEUEGENERICSEND, SEGGER_SYSVIEW_ShrinkId((U32)pxQueue), (U32)pvItemToQueue, xTicksToWait, xCopyPosition)
//#define traceQUEUE_SEND_FROM_ISR( pxQueue )                           SEGGER_SYSVIEW_RecordU32x2(apiID_OFFSET + apiID_XQUEUEGENERICSENDFROMISR, SEGGER_SYSVIEW_ShrinkId((U32)pxQueue), (U32)pxHigherPriorityTaskWoken)
#define traceQUEUE_SEND_FROM_ISR_FAILED( pxQueue )                    SEGGER_SYSVIEW_RecordU32x2(apiID_OFFSET + apiID_XQUEUEGENERICSENDFROMISR, SEGGER_SYSVIEW_ShrinkId((U32)
\end{lstlisting}
The output of the SystemView will then not contain any information about queue events but only visualize task execution times and other FreeRtos calls.
%
%
\section{Percepio Trace Analyzer}
\spic{TickWorks.PNG}{Periodic system ticks when Percepio disabled}{\label{fig:picPeriodicTicks}}%
\spic{AbnormalerTick.PNG}{Irregular system ticks when Percepio enabled}{\label{fig:picIrregularTicks}}%
Just like the SEGGER SystemView, the Percepio Trace Analyzer logs all FreeRtos function calls, including queue operations. To prevent the application from crashing during high capacity due to the additional traffic caused by the Tracelyzer, logging of queue operations can be disabled. To do this, go to the Percepio Trace Processor Expert component and disable the inclusion of queue events.\\
The Percepio Trace component can log data in two ways:
\begin{itemize}
    \item \textbf{Streaming mode:} All log data is transferred from the microcontroller to the Tracelyzer desktop application in real-time
    \item \textbf{Snapshot mode:} There is an on-board buffer that is filled with log data. This buffer can be imported into the Tracelyzer to analyze the performance of the application. No live streaming is possible.
\end{itemize}
When queue events are still included, the amount of data logged by the Tracelyzer has strong negative impacts the performance of the application. No matter if the Percepio Trace component is running in streaming mode or snapshot mode, the application cannot run correctly anymore when more than 1k Byte of data is exchanged between two Serial Switches.\\
During normal operation of the application, the FreeRtos is configured to generate a system tick event every millisecond. Inside the system tick event handler function, an output pin was toggled to verify periodic system ticks and correct FreeRtos execution. The code for the system tick event handler then looks as follows:
\begin{lstlisting}
void FRTOS_vApplicationTickHook(void)
{
    /* Called for every RTOS tick. */
    TMOUT1_AddTick();
    TmDt1_AddTick();
    Pin33_NegVal(); /* toggle Pin 33  on Teensy */
}
\end{lstlisting}
While the Percepio Trace component was disabled, the system tick event was generated periodically every millisecond. This can be verified by looking at the toggling rate of Pin33 on the Teensy. The output was as seen in \autoref{fig:picPeriodicTicks}.\\
When enabling tracing with Percepio by enabling it inside the FreeRtos Processor Expert component, the application was strongly overloaded with logging all queue events. This resulted in unregularities in the system tick events because of blocked FreeRtos calls. The output of Pin 33 on the Teensy can be seen in \autoref{picIrregularTicks}.
The above measurements have been done before the queue events could be disabled inside the Percepio Trace Processor Expert component. Whether disabling the logging of queue events can prevent irregular system ticks is to be evaluated. With enabled queue event logging, the application behavior was the same no matter if the Percepio Trace component was operating in streaming or snapshot mode.\\
Percepio Trace can therefore provide a good indicator for general system performance, task execution times and task intercommunication but is not ideal when lots of FreeRtos functions are used due to the additional traffic it generates.\\
Nevertheless, various inefficiencies were found with thanks to Percepio Trace and it provides good help with finding the cause of Hard Faults because the general area where the application stops is visible in streaming mode.\\
%
%

%
\chapter{Wireless Modems} \label{sec:txtModems}%
% 0_Modems
%
\label{ch:txtModems}
The goal of this project is to create a flexible platform for data routing between devices and modems. But because not all modems behave the same and have equal configuration possibilities, further research needed to be done on the two modems that Aeroscout GmbH plans on using:
\begin{itemize}
    \item RRFD900x
    \item ARF868URL
\end{itemize}
This is also part of the requirements as seen in \autoref{sec:txtAufgabenstellung}.\\
This chapter provides an overview of the configuration possibilities and transmission behavior of both modems.
%
%
%
%
\section{RFD900x} \label{sec:txtRFD900x}
According to he datasheet \cite{RFD900x_Datasheet}, the RFD900x is a long distance radio modem to be integrated into custom projects.\\
Its features include: \begin{itemize}
    \item 3.3V UART interface
    \item The RTS and CTS pins are available to the user
    \item 5V power supply, also via USB
    \item Default serial port settings: 57600 baud, no parity, 8 data bits, 1 stop bit
    \item MAVLink radio status reporting available (RSSI, remote RSSI, local noise, remote noise)
    \item MAVLink protocol framing can be turned on and off.
    \item The RFD900x has two antenna ports and a firmware which supports diversity operation of antennas. During the receive sequence the modem will check both antennas and select the antenna with the best receive signal.
    \item There are three different communication architectures and node topologies selectable: Peer-to-peer, multipoint network and asynchronous non-hopping mesh.
    \item The RFD900x Radio Modem is configurable with methods like the AT Commands and APM Planner.
    \item Golay error correcting code can be enabled (doubles the over-the-air data usage)
    \item Encryption level either off or 128bit 
    \item Adapted version of open source firmware SiK used for the modem
\end{itemize}
Some terms and parameters are explained in more detail below. Generally, this modem is specifically designed for small unmanned vehicles and has multiple features to support this use case.
%
%
\subsection{RTS and CTS Pins}
The modem supports hardware flow control. Ready To Send (RTS) and Clear To Send (CTS) signals are part of the UART communication standard. The transmitter lets the receiver know that it is now ready to transmit data with the RTS line and the receiver lets the transmitter know when it is ready to receive data with the CTS line.
%
%
\subsection{MAVLink Protocol}
MAVLink or Micro Air Vehicle Link is a protocol for communicating with small unmanned vehicle. It is designed as a header-only message library. It is used mostly for communication between a Ground Control Station (GCS) and unmanned vehicles, and in the inter-communication of the subsystem of the vehicle. It can be used to transmit the orientation of the vehicle, its GPS location, speed and many more measurements.
%
%
\subsection{Topology Options}
The modem supports different network setups.
%
\subsubsection{Peer To Peer Network}
\spicv{P2P_Network.png}{Peer To Peer Network}{\label{fig:picP2P_Network}}{70}%
According to \cite{RFD900x_Datasheet}, peer to peer network is a straight forward connection between any two nodes. Whenever  two  nodes  have compatible  parameters  and  are within range, communication will succeed after they synchronize. If your setup requires more than one pair of radios within the same physical pace, you are required to set different  network ID’s to each pair. See \autoref{fig:picP2P_Network}.
%
\subsubsection{Asynchronous Non-Hopping Mesh}
\spicv{AsynchronousNonHoppingMesh.png}{Asynchronous Non-Hopping Mesh}{\label{fig:picAsynchronousNonHoppingMesh}}{90}%
According to \cite{RFD900x_Datasheet}, asynchronous non-hopping mesh is a straight foreward connection between two and more nodes. It allows data transfer across great distances if their settings match. See \autoref{fig:picAsynchronousNonHoppingMesh}.
%
\subsubsection{Multipoint Network}
\spicv{MultipointNetwork.png}{Multipoint Network}{\label{fig:picMultipointNetwork}}{90}%
According to \cite{RFD900x_Datasheet}, in a multipoint connection, the link is between a sender and multiple receivers as long as their configuration matches and they are within range. See \autoref{fig:picMultipointNetwork}.
%
%
\subsection{Configuration Methods}
There are multiple options available on how the RFD900x can be configured.
%
\subsubsection{AT Commands}
The AT command set, formerly known as the Hayes command set, is a command language designed by Dennis Hayes for communication with a 300 baud modem in 1981. The original command set has since been expanded to meet various new needs.\\
In the RFD900x modem, the AT commands can be used to change parameters such as power levels, air data rates, serial speeds etc. The AT command mode can be entered by using the '+++' sequence in a serial terminal connected to the radio. When doing this, the user must allow at least 1 second after any data is sent out to be able to enter the command mode. This prevents the modem to misinterpret the sequence as data to be sent out. The modem will reply with 'OK' as a feedback to the user. Then commands can be entered to set or get modem and transmission parameters.
%
\subsubsection{APM Planner}
APM Planner is an open-source ground station application for MAVlink based autopilots including APM and Pixhawk. It cannot only be used as a mission planner and control application for the autopilot, but it also supports configuration of the connected modems.\\
The APM Planner is an alternative to the QGroundControl used by Aeroscout GmbH.
%
%
\subsection{Error Correcting Code}
The modem can encode the data stream in such a way so that up to 3 bit error in any 24 bit word of encoded data can be recovered and up to 7 bit error can be detected. The applied encoding algorithm is the binary Golay G24.
%
%
\subsection{Encryption}
The 128bit AES data encryption may be set by AT command or any other supported configuration tool. The encryption key can be any 32 character hexadecimal string and less and must be set to the same value on both receiving and sending modem. 
%
%
\subsection{SiK Firmware}
The RFD900x runs with an improved and adapted version of the SiK firmware.\\
The SiK Telemetry Radio is a light and inexpensive open source radio hardware platform that uses open source firmware which has been specially designed to work well with MAVLink packets. It is not only for copters, but also for rovers and planes and is well integrated with the Mission Planner.\\
A SiK Telemetry Radio is one of the easiest ways to setup a telemetry connection between an autopilot (such as Pixhawk or APM) and a ground station (such as QGroundControl).\\
If the ground station supports it, the SiK radios together with the MAVLink protocol can be used to monitor the link quality while flying. This means that the modem can decode the MAVLink protocol and attach radio status parameters like RSSI and noise level to it for the ground station to display to the user.
%
%
%
%
%
%
\section{ARF868URL}
The ARF868 radio modem is a long-distance radio modem with the following features (all taken from the datasheet \cite{ARF_Datasheet}): \begin{itemize}
    \item +-12V RS232 interface
    \item RTS and CTS pins available to the user
    \item 12V power supply
    \item Default serial port settings: 9600 baud, no parity, 8 data bits
    \item MAVLink not decoded
    \item Radio status reports can be retrieved by AT command but not added to MAVLink protocol automatically
    \item One antenna port
    \item There are two different communication architectures and node topologies selectable: Peer-to-peer and multipoint networks
    \item The ARF868 Radio Modem is configurable with the AT command set over serial link and a dedicated configuration software called ``Adeunis RF - Stand Alone Configuration Manager`` which can be downloaded on the Adeunis website
    \item Packet mode can be enabled
    \item Encryption and Golay not supported
\end{itemize}
Most features are similar to the RFD900x and explained above (\autoref{sec:txtRFD900x}). The main difference to this modem is that the ARF868 does not support encryption, Golay error correction and MAVLink decoding. This modem has not specifically been optimized for MAVLink communication and usage within unmanned aeral vehicles.\\
%
%
\subsection{Packet Mode}
According to \cite{ARF_Datasheet}, ARF868 modem uses a packet oriented protocol on its RF interface. The data coming from the UART interface are accumulated in an internal FIFO in the module and then encapsulated in an RF frame. The maximum amount of data that can be transferred in a single radio packet is 1024 Bytes.\\
The maximum packet size can be set up in S218 register from 1 to 1024 bytes. Each new packet introduces some latency in the transmission delay caused by the RF protocol overhead. The RF protocols encapsulate the data payload with the following elements:
\begin{itemize}
    \item  A preamble pattern required for receiver startup time
    \item A bit synchronization pattern to synchronize the receiver on the RF frame
    \item Other protocol field such as source address and destination address, payload length, optional CRC and internal packet type field.
\end{itemize}
The incoming FIFO may accumulate up to 1024 data byte. No more data can be stored in the FIFO while a 1024 bytes block of data has not been released by the radio transmission layer.\\
The user can configure the modem to run in non-secure packet mode where no acknowledges are sent out. The modem can also run in secure packet mode where acknowleges are expected and packages can be retransmitted two times before they are dropped. \\
The RF protocol includes a 16 bit CRC. Each data extracted from an RF packet with an invalid CRC is silently discarded by the state machine module. The CRC ensures that all data received are valid. It can be disabled by the user whose protocols already have a control mechanism integrity or when some bug fixing user protocols are implemented.
%
\section{Conclusion}
The RFD900x is specifically designed for use in unmanned vehicles. It can add information about the link status to the commonly used MAVLink protocol, Golay error correction and detection can be enabled in settings and encrypted communication is also available.\\
The ARF868URL provides no such options but can run in packet mode with a very basic resend behavior configurable.%
%
\chapter{Reliability} \label{sec:txtReliability}
% xx_Reliability
%
The outcome of the software refactoring and the system analysis was an application that runs stable and enough CPU resources left for further features such as encryption and error correcting codes.\\
The main focus of this chapter is therefore the implementation of all features concerning the reliability of the data exchange. As mentioned in the \autoref{sec:txtAufgabenstellung}, Aeroscout GmbH estimates that about 10\% to 20\% of the packages sent are lost and that packages received contain errors due to interference. The application should be expanded to ensure reliability of the data transfer.\\
The implementation of reliability within this project is split into the following parts:
\begin{itemize}
    \item Foreward Error Correction: Error detecting and error correcting code in case of interference
    \item Retransmission: Resend behavior in case of data loss
    \item Algorithm for choosing the optimal modem to (re)send packages
\end{itemize}
%
%
%
%
%
\section{Foreward Error Correction}
The general idea for achieving error detection and correction is to add some redundancy. Error-detection and correction schemes can be either systematic or non-systematic: In a systematic scheme, the transmitter sends the original data, and attaches a fixed number of check bits (or parity data), which are derived from the data bits by some deterministic algorithm. In a system that uses a non-systematic code, the original message is transformed into an encoded message that has at least as many bits as the original message.\\
The aim of encoding a message is to get the content to the recipient with minimal errors. The ground work for error detection and error correction methods has been done by Shannon in the 1940s. Shannon showed that every communication channel can be described by a maximal channel capacity with which information can be exchanged successfully. As long as the transmission rate is smaller or equal to the channel capacity, the transmission error could be arbitrarily small. When redundancy is added, possible errors can be detected or even corrected.
%
%
\subsection{Error Detection}
Error detection is most commonly realized using a suitable hash function (or checksum algorithm). A hash function adds a fixed-length tag to a message, which enables receivers to verify the delivered message by recomputing the tag and comparing it with the one provided. Example: CRC.\cite{ErrorDetectionAndCorrection_Wikipedia}
%
%
\subsection{Error Correction}
An error-correcting code (ECC) or forward error correction (FEC) code is a process of adding redundant data, or parity data, to a message, such that it can be recovered by a receiver even when a number of errors.\\
Error-correcting codes are usually distinguished between convolutional codes and block codes:
\begin{itemize}
    \item Convolutional codes are processed on a bit-by-bit basis. They are particularly suitable for implementation in hardware, and the Viterbi decoder allows optimal decoding.
    \item Block codes are processed on a block-by-block basis. Examples of block codes are repetition codes, Hamming codes, Reed Solomon codes, Golay, BCH and multidimensional parity-check codes. They were followed by a number of efficient codes, Reed–Solomon codes being the most notable due to their current widespread use.
\end{itemize}
%
%
\subsubsection{Reed Solomon}
% https://pdfs.semanticscholar.org/7e94/64b704a9f4b59f9d7df9b437e1b8366b8912.pdf
Reed Solomon is an error-correcting coding system that was devised to address the issue of correcting multiple 
errors, especially burst-type errors in mass storage devices (hard disk drives, DVD, barcode tags), wireless and mobile 
communications units, satellite links, digital TV, digital video broadcasting (DVB), and modem technologies like xDSL. Reed Solomon codes are an important subset of non-binary cyclic error correcting code and are the most widely used codes in practice. These codes are  used in wide range of applications in digital communications and data  storage. \\
Reed Solomon describes a systematic  way of building codes that could detect and correct multiple random symbol errors. By adding t check symbols to the data, an RS code can detect any combination of up to t erroneous symbols, or correct up to $t/2$ symbols. Furthermore, RS codes are suitable as multiple-burst bit-error correcting codes, since a sequence of $b + 1$ consecutive bit errors can affect at most two symbols of size b. The choice of t  is up to the designer of the code, and may be selected within wide limits.\\
RS are  block codes and are  represented as RS (n, k),  where  n is the  size  of code  word length and k is the number of data symbols, $n – k = 2t$ is the number of parity symbols.\\
The properties of Reed-Solomon codes make them especially suited to the applications where burst error occurs. This is because
\begin{itemize}
    \item It does not matter to the code how many bits in a symbol are incorrect, if multiple bits in a symbol are corrupted it only counts as a single error. Alternatively, if a data stream is not characterized by error bursts or drop-outs but by random single bit errors, a Reed-Solomon code is usually a poor choice. More effective cods are available for this case.
    \item Designers are not required to use the natural sizes of Reed-Solomon code blocks. A technique known as shortening produces a smaller code of any desired size from a large code. For example, the widely used (255,251) code can be converted into a (160,128). At the decoder, the same portion of the block is loaded locally with binary zero
    s.
    \item A Reed-Solomon code operating on 8-bits symbols has $n=2^8-1 = 255$ symbols per block because the number of symbol in the encoded block is $n = 2^m-1$
    \item For the designer its capability to correct both burst errors makes it the best choice to use as the encoding and decoding tool.
\end{itemize}
\cite{ReedSolmon_ResearchPaper}
%
%
\subsubsection{Golay}
%Refere5nces:
%  http://aqdi.com/articles/using-the-golay-error-detection-and-correction-code-3/
%  https://www.math.uci.edu/~nckaplan/teaching_files/kaplancodingnotes.pdf
%  https://en.wikipedia.org/wiki/Binary_Golay_code
%  http://cs.indstate.edu/~sbuddha/abstract.pdf      -> anleitung zur codeimplementierung
%  http://www.the-art-of-ecc.com/topics.html         -> codeimplementierung
There are two types of Golay codes: binary golay codes and ternary Golay codes. \\
The binary Golay codes can further be devided into two types: the extended binary Golay code: G24 encodes 12 bits of data in a 24-bit word in such a way that any 3-bit errors can be corrected or any 7-bit errors can be detected, also called the binary (23, 12, 7) quadratic residue (QR) code.\\
The other, the perfect binary Golay code, G23, has codewords of length 23 and is obtained from the extended binary Golay code by deleting one coordinate position (conversely, the extended binary Golay code is obtained from the perfect binary Golay code by adding a parity bit). In standard code notation this code has the parameters [23, 12, 7].
The ternary cyclic code, also known as the G11 code with parameters [11, 6, 5] or G12 with parameters [12, 6, 6] can correct up to 2 errors.
%
%
%
\subsection{Software Implementation of Error Detection and Error Correction}
Because the Reed Solomon error correcting code requires a lot of CPU power and is intended for microcontrollers with more resources, the Golay error correcting code was chosen for this project.\\
Encoding and decoding is done inside the SPI Handler task. The use of the binary Golay error correting code can be enabled per wireless connection by setting the USE\_GOLAY\_ERROR\_CORRECTION to 1 in the configuration file located on the SD card. The Golay source code has been taken from Andrew Tridgell (\cite{GolaySourceCode})) who provides an implementation that can be used without restrictions as long as his copyright is reproduced. This Golay library has also been used in the ArduPilot, an alternative autopilot that is at least as popular as the Pixhawk used by Aeroscout GmbH.\\
The Golay library provides the following interface:
\begin{lstlisting}
/*!
* \fn void golay_encode(uint8_t n, uint8_t* in, uint8_t* out)
* \brief Encodes n bytes of original data into n*2 bytes of encoded data
* \param n: number of bytes to encode, must be multiple of 3
* \param in: pointer to n bytes that will be encoded
* \param out: pointer to memory location where encoded data will be stored
*/
void golay_encode(uint8_t n, uint8_t* in, uint8_t* out);

/*!
* \fn uint8_t golay_decode(uint8_t n, uint8_t* in, uint8_t* out)
* \brief Decodes n bytes of coded data into n/2 bytes of original data
* \param n: number of bytes to decode, must be multiple of 6
* \param in: pointer to n bytes that will be decoded
* \param out: pointer to memory location where decoded data will be stored
* \return number of 12bit words that required correction
*/
uint8_t golay_decode(uint8_t n, uint8_t* in, uint8_t* out);
\end{lstlisting}
Because this implementation uses global variables to save data during encoding and decoding, the library is not reentrant. This is not an issue because the only task using the library is the SPI Handler.\\
The Golay library provides an interface to encode blocks of 3 bytes into blocks of 6 bytes and to decode blocks of 6 bytes into blocks of 3 bytes (see interface from code snippet above). There are two possibilities on how to use the library itself:
\begin{itemize}
    \item Encoding only multiple of 3 bytes and decoding only multiple of 6 bytes, with the risk of delaying some bytes quite long
    \item Adding fill bytes if length of data to encode is not multiple of 3 bytes and adding fill bytes if length of data to decode is not multiple of 6 bytes. This could possibly destroy some code words
\end{itemize}
The option of encoding only multiple of 3 bytes respectively decoding only multiple of 6 bytes has been chosen within this application. Decoding encoded wireless packages then looks as follows:\\
\begin{lstlisting}
/* read byte data from hw buffer */
if(spiSlave == MAX_14830_WIRELESS_SIDE && config.UseGolayPerWlConn[uartNr]) /* read and decode if Golay enabled */
{
    /*
    There's a tradeoff here: the number of data to be decoded needs to be a multiple of 6.
    So we can either just read out as many bytes as there is multiple of 6, risking that we delay some of the bytes quite long.
    Or we can read out all bytes, fill up with pseudo chars and destroy some of the codewords this way.
    => decided to read out only multiples of 6
    */
    if((nofReadBytesToProcess % 6) > 0) /* nof data that will be read is NOT a multiple of six */
    {
        while((nofReadBytesToProcess % 6) > 0)			nofReadBytesToProcess--; /* read out multiples of 6 */
    }
    
    /* read byte data from the HW buffer and decode it */
    spiTransfer(spiSlave, uartNr, MAX_REG_RHR_THR, READ_TRANSFER, encodedBuf, nofReadBytesToProcess); /* read out multiples of 6 */
    nofErrors = golay_decode(nofReadBytesToProcess, &encodedBuf[1], &buffer[1]); /* decode */
    nofReadBytesToProcess = nofReadBytesToProcess / 2; /* Golay doubled the data rate -> after decoding, only half is actual data */
}
\end{lstlisting}
%
\subsubsection{Testing}
The Golay error correcting code doubles the data. But after looking at the system performance in \autoref{subsec:txtPTSystemAnalysis}, this should not be a problem.\\
The application was tested by trying to establish a link between QGroundControl and the autopilot, analogous to \autoref{fig:picQGCSetup}. The link could be established successfully and a periodic heartbeat was exchanged.\\
After consultation with Aeroscout GmbH, it was implied that their main focus does not lie with the error correcting code because they do not expect many bit errors in their transmission but rather full package losses. Therefore no more time was invested in testing the implementation of the Golay algorithm. Before enabling this feature in a final product, more field tests need to be carried out.\\
In fact, the Golay error correcting code can also be enabled with one of the two modems used by Aeroscout GmbH. They reported that they have used a modem with Golay error correction enabled but have not seen a difference in data loss or the communication performance in general, so no more time has been invested in it since then. 
%
%
%
%
\section{Retransmission}
According to Aeroscout GmbH, the most common issue with data exchange is not interference and the resulting bit errors but data loss. Because unmanned aeral vehicles constantly change their position, the data link per modem is not always reliable. Lost packages need to be retransmitted to ensure an uninterrupted data stream. This concept is also known as the Automatic Repeat Request (ARQ). \todo{copy text from arq}\\
Improvements on retransmission behavior have been done during the first software refactoring in this project (see \autoref{sec:txtAddedFeatures}). Package numbering and payload numbering has been introduced, as well as synchronous transmission handling.
%
\chapter{Security} \label{sec:txtSecurity}
% 0_Security
%
The outcome of the software refactoring (see \autoref{sec:txtAddedFeatures}) and the system analysis (see \autoref{sec:SystemAnalysis}) was an application that runs stable and has enough CPU resources left for further features such as encryption and error correcting codes.\\
The main focus of this chapter is therefore the implementation of all features concerning the security of the data exchange. As mentioned in the \autoref{sec:txtAufgabenstellung}, Aeroscout GmbH would like the data exchange between the two Serial Switches to be encrypted and the exchanged data to be stored in a log file that cannot be modified without notice. Therefore, some means of information security has to be implemented.\\
According to \cite{Security_BuEdu}, information security is the practice of preventing unauthorized access, use or modification of information. Its key concepts are: \begin{itemize}
    \item \textbf{Availability:} All systems are functioning correctly and information is available when it is needed.
    \item \textbf{Integrity:} The same data is received as was transmitted, it cannot be modified without detection.
    \item \textbf{Confidentiality:} Data can only be read by the intended receiver. This can be ensured by encrypting the data exchanged so that it is unreadable by anyone who does not have the decryption key.
\end{itemize}
Each key concept and its implementation is elaborated in more detail in this chapter.
%
%
%
%
%
\section{Availability}
Availability of information refers to ensuring that authorized parties are able to access the information when needed.\\
Information only has value if the right people can access the data at the right times.\\
Ensuring availability is not part of the project requirements (see \autoref{sec:txtAufgabenstellung}), it is only listed here for completeness.
%
%
%
%
%
%
%
\section{Integrity}
Integrity of data can be assured with a hash function. A hash is a string or number generated from byte data. The resulting hash is of fixed length, and will vary widely with small variations in input.\\
The only way to recreate the input data from an ideal cryptographic hash function's output is to attempt a brute-force search of possible inputs to see if they produce a match, or use a rainbow table of matched hashes. Therefore, hashing is a one-way function that scrambles plain text to produce a unique message digest.\\
A CRC is an example of a simple hash function and is used to check if the message received matches the message transmitted.\\
Integrity only does not provide security against tempering with the message itself. If someone knows the hash algorithm used, a message can be modified and its hash value recalculated without the receiver knowing about it.
%
\subsection{Implementation of Integrity in Data Exchange}
Data integrity during communication is implemented with a CRC value in the package header and payload to detect bit errors (see \autoref{txtSwOutcomeVa1}). The main microcontroller (K64F) on the Teensy 3.5 has a hardware CRC module that was used for faster CRC calculations. There is a Processor Expert component that uses the hardware CRC module and this component can be configured and used for easier CRC calculation. In the source code, it looks as follows:
\begin{lstlisting}
/* calculate CRC payload */
uint32_t crc16;
CRC1_ResetCRC(CRC1_DeviceData);
CRC1_SetCRCStandard(CRC1_DeviceData, LDD_CRC_MODBUS_16);
CRC1_GetBlockCRC(CRC1_DeviceData, pPackage->payload, pPackage->payloadSize, &crc16);
pPackage->crc16payload = (uint16_t) crc16;

/* calculate crc header */
CRC1_ResetCRC(CRC1_DeviceData);
CRC1_GetCRC8(CRC1_DeviceData, startChar);
CRC1_GetCRC8(CRC1_DeviceData, pPackage->packType);
CRC1_GetCRC8(CRC1_DeviceData, pPackage->devNum);
CRC1_GetCRC8(CRC1_DeviceData, pPackage->sessionNr);
CRC1_GetCRC8(CRC1_DeviceData, *((uint8_t*)(&pPackage->packNr) + 1));
CRC1_GetCRC8(CRC1_DeviceData, *((uint8_t*)(&pPackage->packNr) + 0));
CRC1_GetCRC8(CRC1_DeviceData, *((uint8_t*)(&pPackage->payloadNr) + 1));
CRC1_GetCRC8(CRC1_DeviceData, *((uint8_t*)(&pPackage->payloadNr) + 0));
CRC1_GetCRC8(CRC1_DeviceData, *((uint8_t*)(&pPackage->payloadSize) + 1));
pPackage->crc8Header = CRC1_GetCRC8(CRC1_DeviceData, *((uint8_t*)(&pPackage->payloadSize) + 0));
\end{lstlisting}
The CRC1 component uses the hardware CRC module for accelerated CRC computation.
%
\subsection{Implementation of Integrity for Log Data}
As seen in \autoref{sec:txtAufgabenstellung}, the application must also log the exchanged data and implement means to detect unauthorized modification of the log data. This can be achieved by implementing data integrity for the logged data.
The easiest way to implement integrity in the Logger task is by adding a hash log value for every package logged. The log file then looks similar to this:\\
\noindent\fbox{%
    \parbox{\textwidth}{%
        PackageType;DeviceNumber;SessionNumber;PackageNumber;PayloadNumber;PayloadSize;\\
        CRC8\_Header;Payload;CRC16\_Payload;Hash\\
        \\
        01;00;45;01;01;0011;88;FE092DFF0000000000000608C004039640;AD7C;154987\\
    }
}
The Hash value was simply added after the CRC16\_Payload entry.\\
The problem is that, knowing the hash algorithm used, data can still be tempered with by adding or deleting log lines inside the log file. To detect this misusage, a hash value should not only be calculated for every package but also over the entire log file so that it becomes more difficult to modify log data undetected. But because it would again be easy for malusers to detect the hash algorithm applied, modify the log file and recalculate the hash not only over single log lines but also over the entire modified file, an additional item of security has to be added. Instead of only calculating the hash over the entire file (in addition to the hash calculation over single log lines), a secret random number is also put into the hash algorithm of the entire file before the final hash output is extracted and logged. It lies within the nature of a hash algorithm that the hash output changes drastically with just a small variation of the input value. It is therefore nearly impossible for unauthorized users to find the secret random number used by looking at other hash outputs that were calculated over the entire file.\\
Each K64F microcontroller has a 128-bit unique identification number per chip. This number can be used as the secret random number to be put into the hash algorithm.\\
There is a hardware encryption module on the K64F microcontroller that supports the following hash functions: \begin{itemize}
    \item \textbf{MD5:}\\
    According to \cite{MD5_Wikipedia}, the MD5 algorithm is a widely used hash function producing a 128-bit hash value. Although MD5 was initially designed to be used as a cryptographic hash function, it has been found to suffer from extensive vulnerabilities. It can still be used as a checksum to verify data integrity, but only against unintentional corruption.Like most hash functions, MD5 is neither encryption nor encoding. It can be cracked by brute-force attack and suffers from extensive vulnerabilities.
    \item \textbf{SHA-1:}\\
    According to \cite{SHA1_Wikipedia}, the Secure Hash Algorithm 1 (SHA-1) is a cryptographic hash function which takes an input and produces a 160-bit (20-byte) hash value as an output. Since 2005 SHA-1 has not been considered secure anymore and it is recommended to use SHA-2 or SHA-3 instead.
    \item \textbf{SHA-256:}\\
    The Secure Hash Algorithm 256 (SHA-256) is, just like the SHA-1, a cryptographic hash function. It generates a fixed size 256-bit (32-byte) hash output.
\end{itemize}
%
Although there are no Processor Expert components for either of these hash functions, there is a Crypto Acceleraction Unit (CAU) provided by NXP which is a encryption software library especially designed for the ARM Coretex-M4. The CAU is used for ColdFire and ColdFire+ devices while the mmCAU is for Kinetis devices (ARM Coretex-M4). For more information, consult the user guide and software API in \cite{CAU_UserGuide}.\\
This library can be imported into the Kinetis Design Studio project and used inside the software.
\todo{code snippet on how hash was calculated}
%
%
%
%
%
%
%
\section{Confidentiality}
According to \cite{HashingVsEncrypting}, encryption ensures that only authorized individuals can decipher data. Encryption turns data into a series of unreadable characters, that are not of a fixed length.\\
There are two primary types of encryption: symmetric key encryption and public key encryption.\\
In \textbf{symmetric key encryption}, the key to both encrypt and decrypt is exactly the same. There are numerous standards for symmetric encryption, the popular being AES with a 256 bit key.\\
\textbf{Public key encryption} has two different keys, one used to encrypt data (the public key) and one used to decrypt it (the private key). The public key is made available for anyone to use to encrypt messages, however only the intended recipient has access to the private key, and therefore the ability to decrypt messages.\\
Symmetric encryption provides improved performance, and is simpler to use, however the key needs to be known by both the systems, the one encrypting and the one decrypting data.\\
According to \cite{InformationSecurity_Wikipedia}, information security uses cryptography to transform usable information into a form that renders it unusable by anyone other than an authorized user; this process is called encryption. Information that has been encrypted (rendered unusable) can be transformed back into its original usable form by an authorized user who possesses the cryptographic key, through the process of decryption. 
%
\subsection{Implementation of Encryption}%
The Teensy 3.5 uses the ARM Coretex-M4 microcontroller MK64FX512VMD12. In the MK64FX512xxD12 data sheet (see \cite{NXP_Datasheet}, p.1), it sais that this family supports the following hardware encryption algorithms:\begin{itemize}
    \item \textbf{DES:}\\
    According to \cite{DES_Wikipedia} and \cite{3DES_Wikipedia}, the Data Encryption Standard is a symmetric-key encryption algorithm developed in the early 1970s. It is now considered an insecure encryption standard and can be deciphered quickly, mostly due to its small key size (only 56 bit).  While the small key size was generally sufficient when the algorithm was designed, the increasing computational power made brute-force attacks feasible. 
    \item \textbf{3DES:}\\
    According to \cite{3DES_Wikipedia}, the Triple Data Encryption Standard is a symmetric-key encryption algorithm which applies the DES cipher algorithm three times to each data block. Thereby, all three encryption keys can be identical or independent Theoretically, the resulting key length can be up to 3 x 56 bits = 168 bits. No matter if the keys are independent or identical, the resulting key has a shorter length due to vulnerability to different attacks. 
    \item \textbf{AES:}\\
    According to \cite{AES_Wikipedia}, the Advanced Encryption Standard is, just like the DES, a symmetric-key algorithm and has been established in 2001. It superseded the DES and is now one of the most widely used encryption protocols. Its key sizes can either be 128 bits, 192 bits or 256 bits. 
\end{itemize}
There is a Crypto Acceleraction Unit (CAU) provided by NXP which is a encryption software library especially designed for the ARM Coretex-M4. The CAU is used for ColdFire and ColdFire+ devices while the mmCAU is for Kinetis devices (ARM Coretex-M4). For more information, consult the user guide and software API in \cite{CAU_UserGuide}.\\
Because the RF900x modem already supports encryption and uses the AES algorithm with a 128bit key, this algorithm was also implemented in the application. The encryption key can be stored inside a header file so it is only visible to anyone who has access to the source code. Aeroscout claimed that it is not necessary for them to have a unique encryption key per Serial Switch and that they do not require the key to be changeable during runtime.
%
%
%
%
%
%
%
%
\chapter{Conclusion} \label{sec:txtConclusion}%
% Conclusion
%
The aim of this project was to come up with a flexible application that routes data of connected devices to modems.\\
There was no need to start from scratch for this project as some ground work has been done by Andreas Albisser. He developed a hardware with four RS-232 interfaces to connect data generating and processing devices and four RS-232 interfaces to connect modems for data transmission. The base board has a header to plug in a Teensy 3.2. The Teensy is a small and powerful USB development board that works with Arduino libraries and acts a main micro controller for the base board.\\
The software Andreas Albisser developed for the Teensy 3.2 is complex and not well thought out. Because requirements were added during development, it is hard to maintain and expand.\\
The task description of this project features the use of a more powerful micro controller with Free RTOS as an operating system. Using a different micro controller requires hardware changes. There were two options on how to proceed: either redesign the base board for the new micro controller or design an adapter board that routes the used signals from the new micro controller down to the header of the Teensy 3.2. Because of time reasons, the second option was chosen and an adapter board was designed for the new micro controller used.\\
The Teensy 3.5 was chosen to replace the Teensy 3.2. The Teensy 3.5 features an SD card slot which is also part of the requirements, has more memory and is generally more powerful. Its identical pins are backwards compatible to the pinout of the Teensy 3.2, except for the extra pins because it is slightly longer and has more pins available to the user.\\
The Teensy 3.5 does have hardware debugging pins available on its backside but in order to use them, the on-board bootloader has to be removed as it is in control of the debug pins and cannot be silenced.\\
The hardware debugging pins are routed out on a SWD debug header but the footprint of the header was faulty in the first adapter board version ordered. This mistake was corrected and a second version ordered but the second version had poor inter layer connections. The faulty footprint was tested for correctness and seemed A third version was ordered which is probably better from what can be seen under the microscope. For time reasons, it could not be assembled and tested.\\
Software development was started after the first Teensy Adapter Board had been assembled and the hardware debugging pinout had been corrected manually. First, the software written by Andreas Albisser was analyzed. Because there was almost no documentation available about the software concept or test results, his software concept had to reverse engineered. Because it is complex and hard to expand, a new software concept was drawn up and implemented. The new software concept is according to the ISO/OSI layers and features three main tasks: one for the physical layer (ISO/OSI layer 1), one for the data link layer (ISO/OSI layer 2) and one for the network layer (ISO/OSI layer 3).\\
Layer 1 deals with bytes only and communicates with the hardware components directly. Layer 2 does the assembling of data packages and splits generated data packages into bytes. Layer 3 deals with packages only, their resending in case no acknowledge was received, generates acknowledges for received packages and extracts the payload to push out to the devices connected.\\
Inter task communication is realized with queues, where received data and data to transmit is passed up or down the ISO/OSI layers. All task take the state of all their queues into account during runtime so they will never generate data packages when the queue they should push it to is full. This way, data is never lost intentionally. Currently, each task may lose data unintentionally if a queue operation fails without it being full or empty but for apparently no reason. The only task that will drop data on purpose is the physical layer as it will flush a certain amount of bytes from its queue when full if new data is received and old data has not yet been processed.\\
Generally, the implementation provides about the same functionality as the software developed by Andreas Albisser. The only parameter missing that was part of Andreas' configuration is the limitation of throughput per wireless connection.\\
The base provided by this software is well documented and suitable for further expansion which was not the case with the previous software written for the Teensy 3.2.\\
The next step would be to implement package numbering instead of a system time stamp in the package header. Currently, the receiver does not know about missing packages when looking at the time stamp in the package header because the time stamp is not monotonically increasing. To ensure that packages are always received in the right order, either the sender waits for the acknowledge for a package before sending out the next one or the receiver implements a queue to reassemble packages in their right order before extracting the payload and sending it out to the correct device.\\
Aeroscout GmbH would also like to have data priority configurable. Currently, there is no configuration parameter that allows the user to prioritize one connected device over an other in case of unreliable wireless connection between on-board and off-board Serial Switch.\\
Also, logging has not been implemented yet because of time reasons. The Serial Switch currently prints out debug information on the shell but does not save it in a file.\\
Data encryption and interleaving were part of the task description but have not been implemented yet either.\\
Generally, the aim was of this project was to provide a software that works at least as good as the one provided by Andreas Albisser. Documentation any testing was done at the very end because development of a good and reliable product was prioritized.\\
When testing Andreas Albisser's application with an autopilot software as in a real use-case of Aeroscout GmbH, a connection could not be established successfully when acknowledges were configured for the application and on-board and off-board Serial Switches were communicating wirelessly.\\
When testing the new application with the same autopilot software, a connection was established successfully, even with acknowledges configured and wireless communication between the off-board and on-board devices.\\
Therefore, the goal of this project has been reached because the outcome seems to be of better quality than the state of the project upon start.\\
More time should have been invested into the testing phase to get more detailed results of the project outcome.
%
%
%
%
%
%
%
%
%
\section{Lessons Learned}
This project has been educational in many ways. It was my first time using the operating system Free RTOS and I was struggling during the starting phase because of too many new tools.\\
Also because the start of the project was rather slow and it felt like there was no real progress to show, I was neglecting the documentation up until the very end of the semester. This results in missing pictures in the documentation because they were not taken at that time and missing information because not all the configurations could be remembered when documenting a test case.\\
During the semester, I spent most time working on the project, implementing features and improving performance. This left me with little time for testing at the end where I should have stopped development earlier to invest more time in the testing phase so the person to follow up with this project would know exactly where he/she is at.\\
I do not have a lot of experience with schematics and layouts and did not know about the very high possibility of poor inter layer connections with internally ordered PCBs. Altium Designer together with finding the mistake in the poor inter layer connections cost me a lot of time during this project.\\
Also, I am a first time LaTex user and there were more than one occasion where I remembered Erich Styger's advice: "LaTex without version control is suicide". I can only agree with this statement and am forever glad he mentioned it so early.\\
Last but not least, it will be my goal to always keep the requirements in mind at all times during the next project. It is easy to get lost in coding and adding features. One should always take a step back, look at the requirements and the task description to make sure no element is forgotten. And maybe have a look at the grading paper to see where most time should be invested.%
%
%
%--------------------------------------------------------------------------------------------------------------------
% LITERATURVERZEICHNIS
%--------------------------------------------------------------------------------------------------------------------
\nocite{*} % list all references
\literaturverzeichnis%
{true} % LIteraturverzeichnis anzeigen? (true=ja, false=nein)
%
%
%--------------------------------------------------------------------------------------------------------------------
% BEZEICHNUNGEN
%--------------------------------------------------------------------------------------------------------------------
%
\bezeichnungenChapter%
{true} % Bezeichnungen anzeigen? (true=ja, false=nein)
{%b01_Bezeichnungen
%
%
\bezeichnungenSection{true}{ }{
ACK & Acknowledgement, receiver sends confirmation that package has been received successfully\\
AES & Advanced Encryption Algorithm \\
APM & Arduplane Mission Planer, ground control software for unmanned vehicles \\
ARQ & Automatic Repeat Request \\
CAU & Crypto Acceleration Unit \\
COM port & Simulated serial interface on computer \\
CPU & Central Processing Unit \\
CRC & Cyclic Redundancy Check \\
ECC & Error Correcting Code \\
FatFS & File Allocation Table File System \\
FEC & Foreward Error Correction \\
FIFO & First In First Out, queue style \\
GND & Ground reference, usually 0V\\
GSM & Global System for Mobile communications, telecommunication standard\\
HSLU & Lucerne University of Applied Sciences and Arts\\
HW & Hardware\\
ISO/OSI & 7 Layers Model \\
MCU & Micro Controller Unit \\
NACK & Negative acknowledgement, receiver sends message to sender that a package should be retransmitted\\
PC & Personal Computer\\
PCB & Printed Circuit Board\\
RF & Radio Frequency \\
RS-232 & Serial interface with +-12V \\
RSSI & Received signal strength indicator \\
RTC & Real Time Clock \\
RTT & Real Time Transfer, Segger terminal \\
RTOS & Realtime Operating System \\
RX & Received signal \\
SD & Secure digital, memory card format \\
SHA & Secure Hash Algorithm \\
SNR & Signal to noise ratio \\
SDHC & Secure digital high capacity, memory card format \\
SPI & Serial Peripheral Interface, synchronous communication standard\\
SW & Software\\
SWD & Serial Wire Debug, hardware debugging interface\\
TX & Transmitted signal \\
TTL & Transistor Transistor Level, 5V level\\
UART & Universal Asynchronous Receiver Transmitter \\
UAV & Unmanned Aeral Vehicle \\
UID & Unique ID of the microcontroller \\
USB & Universal Serial Bus
}}



%--------------------------------------------------------------------------------------------------------------------
% ANHANG
%--------------------------------------------------------------------------------------------------------------------

%- - - - - - - - - - - - - - - - - - - - - - - - - - - - - - - - - - - - - - - - - - - - - - - - - - - - - - - - - - 
% outsource_startAppendix
%- - - - - - - - - - - - - - - - - - - - - - - - - - - - - - - - - - - - - - - - - - - - -
\makeatletter 
\def\@makechapterhead#1{
  {\parindent \z@ \raggedright \normalfont
    \interlinepenalty\@M
    \fontsize{16pt}{0pt}\bfseries Anhang \thechapter\quad #1\par\nobreak
    \vskip 60\p@
  }}
\makeatother
%- - - - - - - - - - - - - - - - - - - - - - - - - - - - - - - - - - - - - - - - - - - - -  % Nicht editieren!
%- - - - - - - - - - - - - - - - - - - - - - - - - - - - - - - - - - - - - - - - - - - - - - - - - - - - - - - - - - 

\anhangstuff % Generiert den Anhang-Titel und ändert Spezifikationen
{true} % ist Anhang vorhanden? (true=ja, false=nein)
{
% Anhang content
%
\chapter{A}\label{app:labeInAppendix}
%
}

%- - - - - - - - - - - - - - - - - - - - - - - - - - - - - - - - - - - - - - - - - - - - - - - - - - - - - - - - - - 
% outsource_endAppendix
%- - - - - - - - - - - - - - - - - - - - - - - - - - - - - - - - - - - - - - - - - - - - -
\makeatletter 
\def\@makechapterhead#1{
  {\parindent \z@ \raggedright \normalfont
    \interlinepenalty\@M
    \fontsize{16pt}{0pt}\bfseries \thechapter\quad #1\par\nobreak
    \vskip 60\p@
  }}
\makeatother 
%- - - - - - - - - - - - - - - - - - - - - - - - - - - - - - - - - - - - - - - - - - - - - % Nicht editieren!
%- - - - - - - - - - - - - - - - - - - - - - - - - - - - - - - - - - - - - - - - - - - - - - - - - - - - - - - - - - 


%--------------------------------------------------------------------------------------------------------------------
% LEBENSLAUF (nur in Masterthesis)
%--------------------------------------------------------------------------------------------------------------------
%
%
% Lebenslauf
%
\lebenslauf{true}%
{% Personalien
Name & Stefanie Schmidiger\\
Adresse & Gutenegg\newline
6125 Menzberg\\%
Geburtsdatum & 30.06.1991\\
Heimatort & 6122 Menznau\\
Zivilstand & ledig%
}{% Ausbildung
August 1996 - Juli 2003 & Primarschule, Menzberg\\
August 2003 - Juli 2011 & Kantonsschule, Willisau\\
August 2008 - Juni 2009 & High School Exchange, Plato High School, USA\\
August 2011 - Juli 2013 & Way Up Lehre als Elektronikerin EFZ bei Toradex AG, Horw\\
September 2013 - Juli 2016 & Elektrotechnikstudim Bachelor of Science\newline
Vertiefung Automation \& Embedded Systems\newline
Hochschule Luzern - Technik \& Architektur, Horw\\
Juli 2015 - Januar 2016 & Austauschsemester, Murdoch University, Perth, Australien\\
September 2016 - jetzt & Elektrotechnikstudium Master of Science\newline
Hochschule Luzern - Technik \& Architektur, Horw
}{% Berufliche Tätigkeit
Juli 2003 - August 2003 & Produktionsarbeiten bei Schmidiger GmbH, Menzberg\\
Juli 2004 - August 2004 & Produktionsarbeiten bei Schmidiger GmbH, Menzberg\\
Juli 2005 - August 2005 & Produktionsarbeiten bei Schmidiger GmbH, Menzberg\\
Juli 2006 - August 2006 & Produktionsarbeiten bei Schmidiger GmbH, Menzberg\\
Juli 2007 - August 2007 & Produktionsarbeiten bei Schmidiger GmbH, Menzberg\\
Juli 2009 - August 2009 & Produktionsarbeiten bei Schmidiger GmbH, Menzberg\\
Juli 2010 - August 2010 & Produktionsarbeiten bei Schmidiger GmbH, Menzberg\\
Juli 2014 - August 2014 & Schwimmlehrerin bei Matchpoint Sports Baleares, Mallorca\\
September 2016 - jetzt & Entwicklungsingenieurin bei EVTEC AG, Kriens \\
}%
%
%
%--------------------------------------------------------------------------------------------------------------------
% TODO's (nur für Vorabzüge)
%--------------------------------------------------------------------------------------------------------------------

%\listoftodos%     % Bei der definitiven Ausgabe des Dokuments auskommentieren

%- - - - - - - - - - - - - - - - - - - - - - - - - - - - - - - - - - - - - - - - - - - - - - - - - - - - - - - - - -  
\end{document} % Nicht editieren!
%- - - - - - - - - - - - - - - - - - - - - - - - - - - - - - - - - - - - - - - - - - - - - - - - - - - - - - - - - - 