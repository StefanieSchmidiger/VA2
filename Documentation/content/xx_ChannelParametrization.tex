% Kanalparametrisierung
%
Die wichtigste Eigenschaft eines Empfängers ist seine Empfindlichkeit. Bei digitalen Systemen wird sie über die Bitfehlerrate (BER) bestimmt.\\ Hierzu wird dem Empfänger ein Testsignal mit einer Pseudo-Random-Bitfolge und einem definierten Pegel zugeführt und an seinem Ausgang die Anzahl der Bitfehler gemessen. Das Grundprinzip der BER-Testmodi ist einfach: Der Funkmessplatz sendet einen Datenstrom an das Mobiltelefon, der vom diesem wieder zurückgesendet wird (Loop). Der Messplatz vergleicht gesendete und empfangene Datenströme und ermittelt daraus die Anzahl der Bitfehler. \cite{BitfehlerratenMessungAnGsmMobiltelefonen}\\
