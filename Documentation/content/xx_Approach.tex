% Approach
%
This project builds on the outcome of the project UAV Serial Switch. In this last project, the hardware was designed for the Serial Switch and a software was developed that provides the basic functionalities such as data routing between devices and modems, package transmission on wireless side and the concept of acknowledges and retransmission. For details about the outcome of the last project, please read \autoref{sec:txtSoftwareRefactoring}.\\
The requirements for this follow up project can be found in \autoref{sec:txtAufgabenstellung}.\\
Before improving the software developed in the previous project, the hardware components needed for further software development were ordered and assembled. In the scope of the UAV Serial Switch project, a Teensy adapter board was designed and ordered. But because of manufacturing errors, only one adapter board could be assembled successfully. In order to fully test the application in a real-life setup, two Teensy adapter boards are required. Therefore, more adapter boards were ordered, assembled and tested as a first step within this project.\\
Afterwards, the software developed in the scope of the UAV Serial Switch project was tested with the new hardware. Improvements were made where necessary and new features such as logging were added. Details about the software refactoring can be taken from \autoref{sec:txtSoftwareRefactoring}. Before starting with the implementation of data security and realiability, the runtime behavior of the application was analyzed and assessed (see \autoref{sec:txtSystemAnalysis}). This took up more time than expected because of faulty measurements. It seemed that the CPU was already working at full capacity with no room for the extra traffic that the implementation of security and reliability would cause. Only when finding out that the measurements were faulty due to the extra traffic caused by the analyzation tool could the implementation of the project requirements proceed.\\
Before starting with the implementation of reliability and security, configuration possibilities of the used modems were analyzed (see \autoref{sec:txtModems}).\\ 
Reliability of data exchange was implemented first, starting with an error correcting code. Aeroscout then claimed that their focus lays on retransmission in case of package loss which lead to the testing phase of the error correcting code being cut short.\\
Improvements of the retransmission behavior were done next (see \autoref{sec:txtReliability}), followed by the implementation of encryption (see \autoref{sec:txtSecurity}). Due to a lack of time, both retransmission behavior and encryption could not be implemented fully. Only the concepts and some first implementation steps have been done, elaborations about the detailed implementation and next steps are discussed in this paper.\\
The project plan as implemented can be seen in Appendix \autoref{app:ProjectPlan}.