% Conclusion
%
The goal of this project was to improve the overall performance of the Serial Switch application and to improve reliability of the data exchange and add data security for both transmission and the log files.\\
There was no need to start from scratch for this project as some ground work has been done by Andreas Albisser. He developed a hardware with four RS-232 interfaces to connect data generating and processing devices and four RS-232 interfaces to connect modems for data transmission. The base
board has a header to plug in a Teensy 3.2 development board. Because the Teensy 3.2 has later been replaced by the more powerful and slightly bigger Teensy 3.5 development board, an Adapter Board was designed to map the pins of the Teensy 3.5 to the footprint of the Teensy 3.2. Because of faulty production, only one Teensy Adapter Board was available and more boards were ordered and assembled before starting with any software development.\\
The software developed in the previous project can be configured by changing the parameters inside the config file located on the SD card on the Teensy 3.5. The main feature provided by the software is package transmission on wireless side. Data from connected devices are put into data packages with header and CRC and sent out to connected modems. On the corresponding second Serial Switch, these packages are received and data is extracted and pushed out on device side. Data routing between devices and modems can be configured inside the config file. In the previous project, reliability was ensured by adding configurable retransmission behavior. Acknowledges for data packages can be enabled and when they are, retransmission behavior can be specified inside the config file as well. Retransmission was therefore always predefined by specifying the number of retransmission attempts per wireless connection and the order in which the wireless connections were chosen for retransmission.\\
The software was running with three main tasks that provided the basic functionality of the application and queues as task for task intercommunication:
\begin{itemize}
    \item SPI Handler task, representing physical layer 1 in the ISO-OSI model and handling the hardware interfaces by pushing bytes out and receiving bytes from the hardware components
    \item Package Handler task, representing the data link layer 2 in the ISO-OSI model and assembling packages out of bytes for the Network Handler to process and disassembling packages into bytes for the SPI Handler to process
    \item Network Handler task, representing the network layer 3 and all upper layers in the ISO-OSI model. This task generated data packages for the Package Handler to process and extracted and processed the payload from the assembled packages. The Network Handler also keeps track of packages sent and acknowledges received and handles retransmission in case an acknowledge has not been received.
\end{itemize}
The application was running stable but was still in its first stage and needed refactoring before proceeding with any further implementations. First the task and interrupt priorities were analyzed and package logging was added. To support package reordering in case they get jumbled during transmission, continuous package numbering and continuous payload numbering were introduced. The faulty CRC check was fixed and memory for tasks and queues are now statically allocated.\\
Then the software was analyzed with Percepio Tracelyzer to ensure the application has sufficient CPU resources left for the implementation of encryption. At first, the Percepio Tracelyzer output showed a faulty application where encryption and adding an error correcting code would not be possible. Only after ruling out an error of the FreeRTOS and an error in the application was the bug found with the Percepio Tracelyzer communication. Because of the traffic caused by this analyzation tool, the Serial Switch software was not running stable anymore which lead to the apparently faulty application.\\
Afterwards, some changes were made to the Percepio Tracelyzer interface inside the applications that now reduce the traffic caused by this component. The application was then again analyzed and there are enough CPU resources left for encryption and error detection and error correction.\\
Before starting with the implementation of transmission reliability and data security, the configuration possibilities of the two modems used by Aeroscout GmbH were analyzed. The RFD900x modem is a powerful modem that supports encryption and error correction. It is specifically designed for UAV applications and can also add radio status information to the MAVLink communication protocol used between ground station and UAV. The ARF868 modem is a more generic modem that neither supports encryption nor error correction. It can be configured to run in a packet mode though with acknowledges, CRC check and retransmission of lost packages.\\
Transmission errors can occur either by packages getting lost during transmission or packages containing errors due to interference. This results in different measurements that can be taken by the application to enhance transmission reliability: 
\begin{itemize}
    \item Foreward error correction: Adding an error detection and error correction code so that faulty packages do not need to be retransmitted but the original data can be recovered.
    \item Improving the retransmission behavior
    \item Smart Wireless Selection: Adding an algorithm that dynamically choses the most reliable modem for data transmission
\end{itemize}
The Golay error correction code was chosen because it has also been used by the RFD900x modem. It was implemented and briefly tested but because Aeroscouts expects more packages being lost than faulted, no more time has been invested in testing afterwards.\\
Retransmission behavior was improved by adding an extra task to the software and allowing the application to send out the same packages over multiple wireless interfaces. Due to a lack of time, the software changes implemented within this work package are not done yet but ground work has been done to facilitate future developments and improvements.\\
A Smart Wireless Selection algorithm has been developed but not implemented yet due to a lack of time. That algorithm takes package loss ratio of the last five seconds and a configurable cost value per wireless modem into account.\\
The application has also been prepared for data security but it has not been implemented yet due to a lack of time. There is a crypto library specifically for the microcontroller used which has been added to the Kinetis Design Studio project. It supports the AES encrypton algorithm which would be used for encrypted data communication as this is also the encryption standard that can be enabled on the RFD900x modem.\\
Integrity of the log files can be ensured by using adding a hash value to the log files. But instead of only computing the hash values of the content of the log files, which would be easy to manipulate once the algorithm used is known, a secret value is also put into the hash algorithm. As a secret value, the unique ID of the microcontroller could be used.\\
Now the changes proposed within this document need to be implemented before starting with a next phase. This should not be done in the scope of a student project as the goal of university projects lie within new features and ground work and the next phase of this project consists of software refactoring, adding small changes and improving the overall performance. This is a task that should be handled by Aeroscout itself to make sure they know the strengths and weaknesses of the Serial Switch application as it is.
%
%
%
%
%
%
%
\section{Personal Conclusion}
According to the requirements, the focus of this project lay with data security and reliability of data transmission. But because of endless refactoring possibilities and features that take time to implement, the software refactoring and system analysis took up almost half of the semester. This is the reason why security and integrity remain mostly unfinished and are mostly thought out but not implemented.\\
Generally, someone needs to invest time into the software to have it run stable in every scenario and also implement details and not just the most common use case. This simply cannot be done within the short timeframe of a semester where the focus lay on new features. This is very much a conflict between Aeroscout GmbH who would like a working application at the end of the semester and the HSLU who wants the students to tap into as many unknown fields as possible.